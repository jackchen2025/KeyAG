% -*- coding:UTF-8 -*-
% main.tex
% 主程序文件
\documentclass[UTF8]{ctexart}
\usepackage{amsmath}
\usepackage{amssymb}

\title{解析几何答案}
\author{陈家宝}
\date{}

\bibliographystyle{plain}

\begin{document}

\maketitle
\tableofcontents
\section{向量与坐标}
\subsection{向量的定义、加法和数乘}

\begin{enumerate}
\item 证明:$\overrightarrow{PA_1}+\overrightarrow{PA_2}+ \dots +\overrightarrow{PA_n}=\left(\overrightarrow{OA_1}-\overrightarrow{OP_1}\right)+\left(\overrightarrow{OA_2}-\overrightarrow{OP_2}\right)+ \dots +\left(\overrightarrow{OA_n}-\overrightarrow{OP_n}\right)=\overrightarrow{OA_1}+\overrightarrow{OA_2}+ \dots +\overrightarrow{OA_n}-n\overrightarrow{OP}$,
又$\overrightarrow{OA_1}+\overrightarrow{OA_2}+ \dots +\overrightarrow{OA_n}=\boldsymbol{0}$,
故$\overrightarrow{PA_1}+\overrightarrow{PA_2}+ \dots +\overrightarrow{PA_n}=-n\overrightarrow{OP}=n\overrightarrow{OP}.$

\item 
\begin{enumerate}
\item $\left(\mu - \nu\right)\left(\boldsymbol{a}-\boldsymbol{b}\right)-\left(\mu + \nu\right)\left(\boldsymbol{a}-\boldsymbol{b}\right)=\mu \boldsymbol{a}-\mu \boldsymbol{b}-\nu \boldsymbol{a}+\nu \boldsymbol{b}-\mu \boldsymbol{a}+\mu \boldsymbol{b}-\nu \boldsymbol{a}+\nu \boldsymbol{b}=-2\nu \boldsymbol{a}+2\nu \boldsymbol{b}$.
\item 
$\mathbf{D}=\left|\begin{array}{ccc}
3 & 4\\
2 & -3
\end{array} \right|
=-17,\mathbf{D_x}=\left|\begin{array}{ccc}\mathbf{a}&4\\ \mathbf{b}&-3\end{array}\right| =-3\mathbf{a}-4\mathbf{b},
\mathbf{D_y}=\left|\begin{array}{ccc} 3 & \mathbf{a}\\ 3 & \mathbf{b}\end{array}\right|,\mathbf{x}=\frac{D_x}{D}=\frac{3}{17}\mathbf{a}+\frac{4}{17}\mathbf{b},\mathbf{y}=\frac{D_y}{D}=-\frac{3}{17}\mathbf{b}+\frac{2}{17}\mathbf{a}.$
\end{enumerate}

\item (1)$\mathbf{a}\perp\mathbf{b}$, (2)$\mathbf{a},\mathbf{b}$同向,(3)$\mathbf{a},\mathbf{b}$反向,且$|\mathbf{a}| \geq |\mathbf{b}|$(4)$\mathbf{a},\mathbf{b}$反向,(5)$\mathbf{a},\mathbf{b}$同向,且$|\mathbf{a}| \geq |\mathbf{b}|$.

\item 证明:若$\lambda + \mu < 0, -\lambda < 0 $,由情形1,得$\left[\left(\lambda +\mu\right)+(-\lambda)\right]\mathbf{a}=(\lambda +\mu)\mathbf{a}+(-\lambda)\mathbf{a}$,即$\mu \mathbf{a}=(\lambda +\mu)\mathbf{a}-\lambda \mathbf{a}$,从而$(\lambda +\mu)\mathbf{a} = \lambda \mathbf{a}+\mu \mathbf{a}$得证.
\end{enumerate}

\subsection{向量的线性相关性}
\begin{enumerate}
\item (1)错,当$\mathbf{a}=\mathbf{0}$时;(2)错,当$\mathbf{a}=\mathbf{0}$时.
\item 设$\lambda\mathbf{c}+\mu\mathbf{d}=\mathbf{0},\lambda(2\mathbf{a}-\mathbf{b})+\mu(3\mathbf{a}-2\mathbf{b})=\mathbf{0},(2\lambda+3\mu)\mathbf{a}+(-\lambda-2\mu)\mathbf{b}=\mathbf{0}$. 由于$\mathbf{a},\mathbf{b}$不共线,所以$\left\{\begin{array}{ll}2\lambda+3\mu=0,\\ \lambda+2\mu=0.\end{array}\right.$又$\left|\begin{array}{cc}2&3\\1&2\end{array}\right|=1\neq0$,所以$\lambda=\mu=0$,即$\mathbf{c},\mathbf{d}$线性无关. 
\item 证明:$\overrightarrow{AB}//\overrightarrow{CD}$,$E$、$F$分别为梯形腰$BC$、$AD$上的中点,连接$EF$交$AC$于点$H$,则$H$为$AC$的中点,$\overrightarrow{FH}=\frac{1}{2}\overrightarrow{DC},\overrightarrow{HE}=\frac{1}{2}\overrightarrow{AB},\overrightarrow{FE}=\overrightarrow{FH}+\overrightarrow{HE}=\frac{1}{2}\left(\overrightarrow{DC}+\overrightarrow{AB}\right)$,因为$\overrightarrow{AB}//\overrightarrow{CD}$,而$\overrightarrow{AB}$与$\overrightarrow{CD}$方向一致,所以$\left|\overrightarrow{FE}\right|=\frac{1}{2}\left(\left|\overrightarrow{AB}\right|+\left|\overrightarrow{DC}\right|\right)$.

\item 设$\mathbf{a}=\lambda\mathbf{b}+\mu\mathbf{c}$,则$$\mathbf{a}=-\mathbf{e_1}+3\mathbf{e_2}+2\mathbf{e_3}=2\lambda\mathbf{e_1}-6\lambda\mathbf{e_2}+2\lambda\mathbf{e_3}-3\mu\mathbf{e_1}+12\mu\mathbf{e_2}+11\mu\mathbf{e_3},$$即$$\left(-1-4\lambda+3\mu\right)\mathbf{e_1}+\left(3+6\lambda-12\mu\right)\mathbf{e_2}+\left(2-2\lambda-11\mu\right)\mathbf{e_3}=\mathbf{0},$$又$\mathbf{e_1}$、$\mathbf{e_2}$、$\mathbf{e_3}$线性相关,有$$\left\{\begin{array}{lll}-1-4\lambda+3\mu=0\\ 3+6\lambda-12\mu=0\\ 2-2\lambda-11\mu=0. \end{array}\right.$$解得$\lambda=-\frac{1}{10},\mu=\frac{1}{5}$,所以$\mathbf{a}=\frac{1}{10}\mathbf{b}+\frac{1}{5}\mathbf{c}$.

\item C.

\item 设$\overrightarrow{RD}=\lambda\overrightarrow{AD},\overrightarrow{RE}=\mu\overrightarrow{BE},$则$$\overrightarrow{RD}=\lambda\overrightarrow{AB}+\frac{1}{3}\mu\\ \overrightarrow{BC},\overrightarrow{RE}=\overrightarrow{RD}+\frac{2}{3}\overrightarrow{BC}+\frac{1}{3}\overrightarrow{CA}=\mu \overrightarrow{BC}+\frac{1}{3}\mu \overrightarrow{CA},$$故$$\overrightarrow{RD}=\left(\frac{2}{3}\mu-\frac{1}{3}\right)\overrightarrow{BC}+\frac{1}{3}\left(1-\mu\right)\overrightarrow{AB}=\lambda\overrightarrow{AB}+\frac{1}{3}\lambda\\\overrightarrow{BC},$$推得$$\left\{\begin{array}{ll}\frac{2}{3}\mu-\frac{1}{3}=\frac{1}{3}\lambda,\\ \frac{1}{3}\left(1-\mu\right)=\lambda\end{array}\right.$$解得$$\left\{\begin{array}{ll}\lambda=\frac{1}{7}\\ \mu=\frac{4}{7}\end{array}\right. $$所以$RD=\frac{1}{7}AD,RE=\frac{4}{7}BE$.

\item 由题得$$\overrightarrow{OA}+\overrightarrow{OB}+\overrightarrow{OC}=\overrightarrow{PA}+\overrightarrow{OP}+\overrightarrow{PB}+\overrightarrow{OP}+\overrightarrow{PC}+\overrightarrow{OP}=\left(\overrightarrow{PA}+\overrightarrow{PB}+\overrightarrow{PC}\right)+3\overrightarrow{OP},$$又$\overrightarrow{CP}=2\overrightarrow{PG}=\overrightarrow{PA}+\overrightarrow{PB},$所以$\overrightarrow{PA}+\overrightarrow{PB}+\overrightarrow{PC}=\mathbf{0},$则$\overrightarrow{OA}+\overrightarrow{OB}+\overrightarrow{OC}$得证. 

\item $\overrightarrow{OA}+\overrightarrow{OB}+\overrightarrow{OC}+\overrightarrow{OD}=4\overrightarrow{OP}+\overrightarrow{PA}+\overrightarrow{PB}+\overrightarrow{PC}+\overrightarrow{PD}=4\overrightarrow{OP}+2\overrightarrow{PH}+2\overrightarrow{PG}=4\overrightarrow{OP}$

\item "$\Rightarrow$"因为$A$、$B$、$C$三点共线,所以存在不全为$0$的实数$k$、$l$满足$k\overrightarrow{AB}+l\overrightarrow{AC}=\mathbf{0}$,即$k\left(\overrightarrow{OB}-\overrightarrow{OA}\right)+l\left(\overrightarrow{OC}-\overrightarrow{OA}\right)=\mathbf{0},$化简得$-\left(k+l\right)\overrightarrow{OA}+k\overrightarrow{OB}+l\overrightarrow{OC}=\mathbf{0},$分别取$\lambda=-\left(k+l\right),\mu=k,\gamma=l,$得证.\\
"$\Leftarrow$"因为$\lambda=-\left(\mu+\gamma\right)$,设$\lambda\neq0$,则$\mu$、$\gamma$不全为$0$,$-\left(\mu+\gamma\right)\overrightarrow{OA}+\mu\overrightarrow{OB}+\gamma\overrightarrow{OC}=\mathbf{0}$,化简得$\mu\left(\overrightarrow{OB}-\overrightarrow{OA}\right)+\gamma\left(\overrightarrow{OC}-\overrightarrow{OA}\right)=\mathbf{0}$,即$\mu\overrightarrow{AB}+\gamma\overrightarrow{AC}=\mathbf{0}$,故$A$、$B$、$C$三点共线. 

\item "$\Rightarrow$"因为$P_1,P_2,P_3,P_4$四点共面,所以$\overrightarrow{P_{1}P_{2}},\overrightarrow{P_{1}P_{3}},\overrightarrow{P_{1}P_{4}}$线性相关,存在不全为$0$的$m,n,p$使得$m\overrightarrow{P_{1}P_{2}}+n\overrightarrow{P_{1}P_{3}}+p\overrightarrow{P_{1}P_{4}}=\mathbf{0},$即$$m\left(\overrightarrow{OP_2}-\overrightarrow{OP_1}\right)+n\left(\overrightarrow{OP_3}-\overrightarrow{OP_1}\right)+p\left(\overrightarrow{OP_4}-\overrightarrow{OP_1}\right)=\mathbf{0},$$即$$-\left(m+n+p\right)\mathbf{n}+m\mathbf{r_2}+n\mathbf{r_3}+p\mathbf{r_4}=\mathbf{0},$$令$m+n+p=\lambda_1,\lambda_2=m,\lambda_3=n,\lambda_4=p$,得证. \\
"$\Leftarrow$"设$\lambda_1\neq0$,则$\lambda_1=-\left(\lambda_2+\lambda_3+\lambda_4\right)$,所以$\lambda_2,\lambda_3,\lambda_4$不全为$0$,$$-\left(\lambda_2+\lambda_3+\lambda_4\right)\mathbf{r_1}+\lambda_2\mathbf{r_2}+\lambda_3\mathbf{r_3}+\lambda_4\mathbf{r_4}=\mathbf{0},$$因此$P_1,P_2,P_3,P_4$四点共面. 

\item $A,B,C$三点不共线$\Leftrightarrow$$\overrightarrow{AB},\overrightarrow{AC}$不共线$\Leftrightarrow$点$P$在$\pi$上$\Leftrightarrow$$\overrightarrow{AP}=\mu\overrightarrow{AB}+\gamma\overrightarrow{AC}\left(\mu,\gamma\in \mathbb{R}\right)$$\Leftrightarrow$$\overrightarrow{OP}-\overrightarrow{OA}=\mu\left(\overrightarrow{OB}-\overrightarrow{OA}\right)+\gamma\left(\overrightarrow{OC}-\overrightarrow{OA}\right)$$\Leftrightarrow$$\overrightarrow{OP}=\left(1-\mu-\gamma\right)\overrightarrow{OA}+\mu\overrightarrow{OB}+\gamma\overrightarrow{OC},$取$\gamma=1-\mu-\gamma$,得证. 

\item (1)$\overrightarrow{AD}=\frac{2}{3}\mathbf{e_1}+\frac{1}{3}\mathbf{e_2},\overrightarrow{AE}=\frac{1}{3}\mathbf{e_1}+\frac{2}{3}\mathbf{e_2}$\\
(2)由角平分线的性质得$\frac{\left|\overrightarrow{BT}\right|}{\left|\overrightarrow{TC}\right|}=\frac{\mathbf{e_1}}{\mathbf{e_2}}$,又$\overrightarrow{BT}$与$\overrightarrow{TC}$同向,则$\overrightarrow{BT}=\frac{\mathbf{e_1}}{\mathbf{e_2}}\overrightarrow{TC},\overrightarrow{BT}=\overrightarrow{AT}-\overrightarrow{AB},\overrightarrow{TC}=\overrightarrow{AC}-\overrightarrow{AT}$,因此$\overrightarrow{AT}-\overrightarrow{AB}=\frac{\mathbf{e_1}}{\mathbf{e_2}}\left(\overrightarrow{AC}-\overrightarrow{AT}\right)$,得$\overrightarrow{AT}=\frac{\left|e_1\right|+\left|e_2\right|}{\left|e_1\right|+\left|e_2\right|}\mathbf{e_1}$. 
\end{enumerate}

\subsection{标架与坐标}
\begin{enumerate}
\item (1)$\left(0,16,-1\right).$(2)$\left(-11,9,-2\right).$

\item 分析:以本书第25页推论1.6.1作判别式,以本书第7页定理1.21(4)\\
(1)$\left(\mathbf{a},\mathbf{b},\mathbf{c}\right)=\begin{array}{ccc}5&2&1\\-1&4&2\\-1&-1&5\end{array}=121\neq0,$故$\mathbf{a},\mathbf{b},\mathbf{c}$不共面,无线性组合. \\
(2)同理$\mathbf{a},\mathbf{b},\mathbf{c}$共面,\\
设$\mathbf{c}=\lambda\mathbf{a}+\mu\mathbf{b},$\\
得$\left\{\begin{array}{lll}-3=6\lambda-9\mu\\6=4\lambda+6\mu\\ 3=2\lambda-3\mu\end{array}\right.$\\
解得$\mathbf{c}=\frac{1}{2}\mathbf{a}+\frac{4}{3}\mathbf{b}.$
(3)同理$\mathbf{a},\mathbf{b},\mathbf{c}$共面,\\
但$\mathbf{a}$平行$\mathbf{b}$,且$\mathbf{a}\neq\mathbf{c},$故显然无法以线性组合表示$\mathbf{c}$. 

\item 证明:设四面体$A_1A_2A_3A_4$中,$A_i$所对得面的重心为$G_i$,\\
欲证$A_{i}G_{i}\left(i=1,2,3,4\right)$相交于一点,在$A_{i}G_{i}$上取一点$P_{i}$使得$\overrightarrow{A_{i}G_{i}}=3\overrightarrow{P_{i}G_{i}},$\\ 
从而$\overrightarrow{OP_{i}}=\frac{\overrightarrow{OA_{i}}+3\overrightarrow{OG_{i}}}{4},$\\
设$A_{i}$坐标为$\left(x_i,y_i,z_i\right)\left(i=1,2,3,4\right)$则有
$$G_{1}\left(\frac{x_2+x_3+x_4}{3},\frac{y_2+y_3+y_4}{3},\frac{z_2+z_3+z_4}{3}\right),$$
$$G_{2}\left(\frac{x_1+x_3+x_4}{3},\frac{y_1+y_3+y_4}{3},\frac{z_1+z_3+z_4}{3}\right),$$
$$G_{3}\left(\frac{x_1+x_2+x_4}{3},\frac{y_1+y_2+y_4}{3},\frac{z_1+z_2+z_4}{3}\right),$$
$$G_{4}\left(\frac{x_1+x_2+x_3}{3},\frac{y_1+y_2+y_3}{3},\frac{z_1+z_2+z_3}{3}\right),$$
所以$P_{1}\left(\frac{x_1+3\frac{x_2+x_3+x_4}{3}}{4},\frac{y_1+3\frac{y_2+y_3+y_4}{3}}{4},\frac{z_1+3\frac{z_2+z_3+z_4}{3}}{4}\right),$
即$P_{1}\left(\frac{x_1+x_2+x_3+x_4}{4},\frac{y_1+y_2+y_3+y_4}{4},\frac{z_1+z_2+z_3+z_4}{4}\right),$
同理可得$P_{2},P_{3},P_{4}$坐标,可知$P_{1},P_{2},P_{3},P_{4}$为同一点,故$A_{i}G_{i}$交于同一点$P$且点$P$到任一顶点的距离等于此点到对面重心的三倍. 

\item 证明:必要性:因为$\pi$上三点$p_{i}\left(x_i,y_i\right)_{i=1,2,3}$共线,故$\overrightarrow{p_1p_2}$平行于$\overrightarrow{p_1p_3}$,即$\frac{x_2-x_1}{x_3-x_1}=\frac{y_2-y_1}{y_3-y_1}$\\
即$x_1y_2+x_2y_3+x_3y_1-x_1y_3-x_3y_2-x_2y_1=\left|\begin{array}{ccc}x_1&y_1&1\\x_2&y_2&1\\x_3&y_3&1\end{array}\right|=0$.
充分性:由$\left|\begin{array}{ccc}x_1&y_1&1\\x_2&y_2&1\\x_3&y_3&1\end{array}\right|=x_1y_2+x_2y_3+x_3y_1-x_1y_3-x_3y_2-x_2y_1=0$整理得$$\frac{x_2-x_1}{x_3-x_1}=\frac{y_2-y_1}{y_3-y_1},$$
即$\overrightarrow{p_1p_2}$平行于$\overrightarrow{p_1p_3}$,所以$\pi$上三点$p_{i}\left(x_i,y_i\right)_{i=1,2,3}$共线. 
综上,$\pi$上三点$p_{i}\left(x_i,y_i\right)_{i=1,2,3}$共线当且仅当$\left|\begin{array}{ccc}x_1&y_1&1\\x_2&y_2&1\\x_3&y_3&1\end{array}\right|=0.$

\item 证明:建立仿射坐标系$\left\{\overrightarrow{A},\overrightarrow{AB},\overrightarrow{AC}\right\},$由$\overrightarrow{AP}=\lambda\overrightarrow{PB}=\lambda\left(\overrightarrow{AB}-\overrightarrow{AP}\right),$得$\overrightarrow{AP}=\frac{\lambda}{\lambda+1}\overrightarrow{AB},\overrightarrow{AP}=\left(\frac{\lambda}{\lambda+1},0\right)$;\\
$\overrightarrow{AR}=\frac{1}{1+\nu}\overrightarrow{AC},\overrightarrow{AR}=\left(0,\frac{1}{1+\nu}\right)$;\\
$\overrightarrow{AQ}=\frac{1}{1+\mu}\overrightarrow{AB}+\frac{\mu}{1+\mu}\overrightarrow{AC},\overrightarrow{AQ}=\left(\frac{1}{1+\mu},\frac{\mu}{1+\mu}\right)$;\\
由$P,Q,R$共线当且仅当$\left|\begin{array}{ccc}\frac{1}{1+\mu}&0&1\\ 0&\frac{1}{1+\nu}&1\\ \frac{1}{1+\mu}&\frac{\mu}{1+\mu}&1\end{array}\right|=0$,得$\lambda\mu\nu=-1$,证毕. \\
(注:事实上,此即平面几何上的梅涅劳斯定理)
\end{enumerate}

\subsection{数量积}

\bibliography{math}
\end{document}