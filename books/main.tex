% -*- coding:UTF-8 -*-
% main.tex
% 主程序文件
\documentclass[UTF8]{ctexart}
\usepackage{amsmath}
\usepackage{amssymb}
\usepackage{enumerate}
\usepackage{TikZ}


\title{解析几何答案}
\author{陈家宝\\jackchen2025@outlook.com}
\date{\today}

\bibliographystyle{plain}

\begin{document}

\maketitle
\tableofcontents

\section{向量与坐标}
\subsection{向量的定义、加法和数乘}

\begin{enumerate}
\item 证明:$\overrightarrow{PA_1}+\overrightarrow{PA_2}+ \dots +\overrightarrow{PA_n}=\left(\overrightarrow{OA_1}-\overrightarrow{OP_1}\right)+\left(\overrightarrow{OA_2}-\overrightarrow{OP_2}\right)+ \dots +\left(\overrightarrow{OA_n}-\overrightarrow{OP_n}\right)=\overrightarrow{OA_1}+\overrightarrow{OA_2}+ \dots +\overrightarrow{OA_n}-n\overrightarrow{OP}$,
又$\overrightarrow{OA_1}+\overrightarrow{OA_2}+ \dots +\overrightarrow{OA_n}=\boldsymbol{0}$,
故$\overrightarrow{PA_1}+\overrightarrow{PA_2}+ \dots +\overrightarrow{PA_n}=-n\overrightarrow{OP}=n\overrightarrow{OP}.$

\item 
\begin{enumerate}[(1)]
\item $\left(\mu - \upsilon\right)\left(\boldsymbol{a}-\boldsymbol{b}\right)-\left(\mu + \upsilon\right)\left(\boldsymbol{a}-\boldsymbol{b}\right)=\mu \boldsymbol{a}-\mu \boldsymbol{b}-\upsilon \boldsymbol{a}+\upsilon \boldsymbol{b}-\mu \boldsymbol{a}+\mu \boldsymbol{b}-\upsilon \boldsymbol{a}+\upsilon \boldsymbol{b}=-2\upsilon \boldsymbol{a}+2\upsilon \boldsymbol{b}$.
\item 
$\mathbf{D}=\left|\begin{array}{ccc}
3 & 4\\
2 & -3
\end{array} \right|
=-17,\mathbf{D_x}=\left|\begin{array}{ccc}\mathbf{a}&4\\ \mathbf{b}&-3\end{array}\right| =-3\mathbf{a}-4\mathbf{b},
\mathbf{D_y}=\left|\begin{array}{ccc} 3 & \mathbf{a}\\ 3 & \mathbf{b}\end{array}\right|,\mathbf{x}=\displaystyle\frac{D_x}{D}=\displaystyle\frac{3}{17}\mathbf{a}+\displaystyle\frac{4}{17}\mathbf{b},\mathbf{y}=\displaystyle\frac{D_y}{D}=-\displaystyle\frac{3}{17}\mathbf{b}+\displaystyle\frac{2}{17}\mathbf{a}.$
\end{enumerate}

\item (1)$\mathbf{a}\perp\mathbf{b}$, (2)$\mathbf{a},\mathbf{b}$同向,(3)$\mathbf{a},\mathbf{b}$反向,且$|\mathbf{a}| \geq |\mathbf{b}|$(4)$\mathbf{a},\mathbf{b}$反向,(5)$\mathbf{a},\mathbf{b}$同向,且$|\mathbf{a}| \geq |\mathbf{b}|$.

\item 证明:若$\lambda + \mu < 0, -\lambda < 0 $,由情形1,得$\left[\left(\lambda +\mu\right)+(-\lambda)\right]\mathbf{a}=(\lambda +\mu)\mathbf{a}+(-\lambda)\mathbf{a}$,即$\mu \mathbf{a}=(\lambda +\mu)\mathbf{a}-\lambda \mathbf{a}$,从而$(\lambda +\mu)\mathbf{a} = \lambda \mathbf{a}+\mu \mathbf{a}$得证.
\end{enumerate}

\subsection{向量的线性相关性}
\begin{enumerate}
\item (1)错,当$\mathbf{a}=\mathbf{0}$时;(2)错,当$\mathbf{a}=\mathbf{0}$时.
\item 设$\lambda\mathbf{c}+\mu\mathbf{d}=\mathbf{0},\lambda(2\mathbf{a}-\mathbf{b})+\mu(3\mathbf{a}-2\mathbf{b})=\mathbf{0},(2\lambda+3\mu)\mathbf{a}+(-\lambda-2\mu)\mathbf{b}=\mathbf{0}$. 由于$\mathbf{a},\mathbf{b}$不共线,所以$\left\{\begin{array}{ll}2\lambda+3\mu=0,\\ \lambda+2\mu=0.\end{array}\right.$又$\left|\begin{array}{cc}2&3\\1&2\end{array}\right|=1\neq0$,所以$\lambda=\mu=0$,即$\mathbf{c},\mathbf{d}$线性无关. 
\item 证明:$\overrightarrow{AB}//\overrightarrow{CD}$,$E$、$F$分别为梯形腰$BC$、$AD$上的中点,连接$EF$交$AC$于点$H$,则$H$为$AC$的中点,$\overrightarrow{FH}=\displaystyle\frac{1}{2}\overrightarrow{DC},\overrightarrow{HE}=\displaystyle\frac{1}{2}\overrightarrow{AB},\overrightarrow{FE}=\overrightarrow{FH}+\overrightarrow{HE}=\displaystyle\frac{1}{2}\left(\overrightarrow{DC}+\overrightarrow{AB}\right)$,因为$\overrightarrow{AB}//\overrightarrow{CD}$,而$\overrightarrow{AB}$与$\overrightarrow{CD}$方向一致,所以$\left|\overrightarrow{FE}\right|=\displaystyle\frac{1}{2}\left(\left|\overrightarrow{AB}\right|+\left|\overrightarrow{DC}\right|\right)$.

\item 设$\mathbf{a}=\lambda\mathbf{b}+\mu\mathbf{c}$,则$$\mathbf{a}=-\mathbf{e_1}+3\mathbf{e_2}+2\mathbf{e_3}=2\lambda\mathbf{e_1}-6\lambda\mathbf{e_2}+2\lambda\mathbf{e_3}-3\mu\mathbf{e_1}+12\mu\mathbf{e_2}+11\mu\mathbf{e_3},$$即$$\left(-1-4\lambda+3\mu\right)\mathbf{e_1}+\left(3+6\lambda-12\mu\right)\mathbf{e_2}+\left(2-2\lambda-11\mu\right)\mathbf{e_3}=\mathbf{0},$$又$\mathbf{e_1}$、$\mathbf{e_2}$、$\mathbf{e_3}$线性相关,有$$\left\{\begin{array}{lll}-1-4\lambda+3\mu=0\\ 3+6\lambda-12\mu=0\\ 2-2\lambda-11\mu=0. \end{array}\right.$$解得$\lambda=-\displaystyle\frac{1}{10},\mu=\displaystyle\frac{1}{5}$,所以$\mathbf{a}=\displaystyle\frac{1}{10}\mathbf{b}+\displaystyle\frac{1}{5}\mathbf{c}$.

\item C.

\item 设$\overrightarrow{RD}=\lambda\overrightarrow{AD},\overrightarrow{RE}=\mu\overrightarrow{BE},$则$$\overrightarrow{RD}=\lambda\overrightarrow{AB}+\displaystyle\frac{1}{3}\mu\\ \overrightarrow{BC},\overrightarrow{RE}=\overrightarrow{RD}+\displaystyle\frac{2}{3}\overrightarrow{BC}+\displaystyle\frac{1}{3}\overrightarrow{CA}=\mu \overrightarrow{BC}+\displaystyle\frac{1}{3}\mu \overrightarrow{CA},$$故$$\overrightarrow{RD}=\left(\displaystyle\frac{2}{3}\mu-\displaystyle\frac{1}{3}\right)\overrightarrow{BC}+\displaystyle\frac{1}{3}\left(1-\mu\right)\overrightarrow{AB}=\lambda\overrightarrow{AB}+\displaystyle\frac{1}{3}\lambda\\\overrightarrow{BC},$$推得$$\left\{\begin{array}{ll}\displaystyle\frac{2}{3}\mu-\displaystyle\frac{1}{3}=\displaystyle\frac{1}{3}\lambda,\\ \displaystyle\frac{1}{3}\left(1-\mu\right)=\lambda\end{array}\right.$$解得$$\left\{\begin{array}{ll}\lambda=\displaystyle\frac{1}{7}\\ \mu=\displaystyle\frac{4}{7}\end{array}\right. $$所以$RD=\displaystyle\frac{1}{7}AD,RE=\displaystyle\frac{4}{7}BE$.

\item 由题得$$\overrightarrow{OA}+\overrightarrow{OB}+\overrightarrow{OC}=\overrightarrow{PA}+\overrightarrow{OP}+\overrightarrow{PB}+\overrightarrow{OP}+\overrightarrow{PC}+\overrightarrow{OP}=\left(\overrightarrow{PA}+\overrightarrow{PB}+\overrightarrow{PC}\right)+3\overrightarrow{OP},$$又$\overrightarrow{CP}=2\overrightarrow{PG}=\overrightarrow{PA}+\overrightarrow{PB},$所以$\overrightarrow{PA}+\overrightarrow{PB}+\overrightarrow{PC}=\mathbf{0},$则$\overrightarrow{OA}+\overrightarrow{OB}+\overrightarrow{OC}$得证. 

\item $\overrightarrow{OA}+\overrightarrow{OB}+\overrightarrow{OC}+\overrightarrow{OD}=4\overrightarrow{OP}+\overrightarrow{PA}+\overrightarrow{PB}+\overrightarrow{PC}+\overrightarrow{PD}=4\overrightarrow{OP}+2\overrightarrow{PH}+2\overrightarrow{PG}=4\overrightarrow{OP}$

\item "$\Rightarrow$"因为$A$、$B$、$C$三点共线,所以存在不全为$0$的实数$k$、$l$满足$k\overrightarrow{AB}+l\overrightarrow{AC}=\mathbf{0}$,即$k\left(\overrightarrow{OB}-\overrightarrow{OA}\right)+l\left(\overrightarrow{OC}-\overrightarrow{OA}\right)=\mathbf{0},$化简得$-\left(k+l\right)\overrightarrow{OA}+k\overrightarrow{OB}+l\overrightarrow{OC}=\mathbf{0},$分别取$\lambda=-\left(k+l\right),\mu=k,\gamma=l,$得证.\\
"$\Leftarrow$"因为$\lambda=-\left(\mu+\gamma\right)$,设$\lambda\neq0$,则$\mu$、$\gamma$不全为$0$,$-\left(\mu+\gamma\right)\overrightarrow{OA}+\mu\overrightarrow{OB}+\gamma\overrightarrow{OC}=\mathbf{0}$,化简得$\mu\left(\overrightarrow{OB}-\overrightarrow{OA}\right)+\gamma\left(\overrightarrow{OC}-\overrightarrow{OA}\right)=\mathbf{0}$,即$\mu\overrightarrow{AB}+\gamma\overrightarrow{AC}=\mathbf{0}$,故$A$、$B$、$C$三点共线. 

\item "$\Rightarrow$"因为$P_1,P_2,P_3,P_4$四点共面,所以$\overrightarrow{P_{1}P_{2}},\overrightarrow{P_{1}P_{3}},\overrightarrow{P_{1}P_{4}}$线性相关,存在不全为$0$的$m,n,p$使得$m\overrightarrow{P_{1}P_{2}}+n\overrightarrow{P_{1}P_{3}}+p\overrightarrow{P_{1}P_{4}}=\mathbf{0},$即$$m\left(\overrightarrow{OP_2}-\overrightarrow{OP_1}\right)+n\left(\overrightarrow{OP_3}-\overrightarrow{OP_1}\right)+p\left(\overrightarrow{OP_4}-\overrightarrow{OP_1}\right)=\mathbf{0},$$即$$-\left(m+n+p\right)\mathbf{n}+m\mathbf{r_2}+n\mathbf{r_3}+p\mathbf{r_4}=\mathbf{0},$$令$m+n+p=\lambda_1,\lambda_2=m,\lambda_3=n,\lambda_4=p$,得证. \\
"$\Leftarrow$"设$\lambda_1\neq0$,则$\lambda_1=-\left(\lambda_2+\lambda_3+\lambda_4\right)$,所以$\lambda_2,\lambda_3,\lambda_4$不全为$0$,$$-\left(\lambda_2+\lambda_3+\lambda_4\right)\mathbf{r_1}+\lambda_2\mathbf{r_2}+\lambda_3\mathbf{r_3}+\lambda_4\mathbf{r_4}=\mathbf{0},$$因此$P_1,P_2,P_3,P_4$四点共面. 

\item $A,B,C$三点不共线$\Leftrightarrow$$\overrightarrow{AB},\overrightarrow{AC}$不共线$\Leftrightarrow$点$P$在$\pi$上$\Leftrightarrow$$\overrightarrow{AP}=\mu\overrightarrow{AB}+\gamma\overrightarrow{AC}\left(\mu,\gamma\in \mathbb{R}\right)$$\Leftrightarrow$$\overrightarrow{OP}-\overrightarrow{OA}=\mu\left(\overrightarrow{OB}-\overrightarrow{OA}\right)+\gamma\left(\overrightarrow{OC}-\overrightarrow{OA}\right)$$\Leftrightarrow$$\overrightarrow{OP}=\left(1-\mu-\gamma\right)\overrightarrow{OA}+\mu\overrightarrow{OB}+\gamma\overrightarrow{OC},$取$\gamma=1-\mu-\gamma$,得证. 

\item (1)$\overrightarrow{AD}=\displaystyle\frac{2}{3}\mathbf{e_1}+\displaystyle\frac{1}{3}\mathbf{e_2},\overrightarrow{AE}=\displaystyle\frac{1}{3}\mathbf{e_1}+\displaystyle\frac{2}{3}\mathbf{e_2}$\\
(2)由角平分线的性质得$\displaystyle\frac{\left|\overrightarrow{BT}\right|}{\left|\overrightarrow{TC}\right|}=\displaystyle\frac{\mathbf{e_1}}{\mathbf{e_2}}$,又$\overrightarrow{BT}$与$\overrightarrow{TC}$同向,则$\overrightarrow{BT}=\displaystyle\frac{\mathbf{e_1}}{\mathbf{e_2}}\overrightarrow{TC},\overrightarrow{BT}=\overrightarrow{AT}-\overrightarrow{AB},\overrightarrow{TC}=\overrightarrow{AC}-\overrightarrow{AT}$,因此$\overrightarrow{AT}-\overrightarrow{AB}=\displaystyle\frac{\mathbf{e_1}}{\mathbf{e_2}}\left(\overrightarrow{AC}-\overrightarrow{AT}\right)$,得$\overrightarrow{AT}=\displaystyle\frac{\left|e_1\right|+\left|e_2\right|}{\left|e_1\right|+\left|e_2\right|}\mathbf{e_1}$. 
\end{enumerate}

\subsection{标架与坐标}
\begin{enumerate}
\item (1)$\left(0,16,-1\right).$(2)$\left(-11,9,-2\right).$

\item 分析:以本书第25页推论1.6.1作判别式,以本书第7页定理1.21(4)\\
\begin{enumerate}[(1)]
\item $\left(\mathbf{a},\mathbf{b},\mathbf{c}\right)=\left|\begin{array}{ccc}5&2&1\\-1&4&2\\-1&-1&5\end{array}\right|=121\neq0,$故$\mathbf{a},\mathbf{b},\mathbf{c}$不共面,无线性组合. \\
\item 同理$\mathbf{a},\mathbf{b},\mathbf{c}$共面,\\
设$\mathbf{c}=\lambda\mathbf{a}+\mu\mathbf{b},$\\
得$\left\{\begin{array}{lll}-3=6\lambda-9\mu\\6=4\lambda+6\mu\\ 3=2\lambda-3\mu\end{array}\right.$\\
解得$\mathbf{c}=\displaystyle\frac{1}{2}\mathbf{a}+\displaystyle\frac{4}{3}\mathbf{b}.$
\item 同理$\mathbf{a},\mathbf{b},\mathbf{c}$共面,
但$\mathbf{a}$平行$\mathbf{b}$,且$\mathbf{a}\neq\mathbf{c},$故显然无法以线性组合表示$\mathbf{c}$. 
\end{enumerate}
\item 证明:设四面体$A_1A_2A_3A_4$中,$A_i$所对得面的重心为$G_i$,\\
欲证$A_{i}G_{i}\left(i=1,2,3,4\right)$相交于一点,在$A_{i}G_{i}$上取一点$P_{i}$使得$\overrightarrow{A_{i}G_{i}}=3\overrightarrow{P_{i}G_{i}},$\\ 
从而$\overrightarrow{OP_{i}}=\displaystyle\frac{\overrightarrow{OA_{i}}+3\overrightarrow{OG_{i}}}{4},$\\
设$A_{i}$坐标为$\left(x_i,y_i,z_i\right)\left(i=1,2,3,4\right)$则有
$$G_{1}\left(\displaystyle\frac{x_2+x_3+x_4}{3},\displaystyle\frac{y_2+y_3+y_4}{3},\displaystyle\frac{z_2+z_3+z_4}{3}\right),$$
$$G_{2}\left(\displaystyle\frac{x_1+x_3+x_4}{3},\displaystyle\frac{y_1+y_3+y_4}{3},\displaystyle\frac{z_1+z_3+z_4}{3}\right),$$
$$G_{3}\left(\displaystyle\frac{x_1+x_2+x_4}{3},\displaystyle\frac{y_1+y_2+y_4}{3},\displaystyle\frac{z_1+z_2+z_4}{3}\right),$$
$$G_{4}\left(\displaystyle\frac{x_1+x_2+x_3}{3},\displaystyle\frac{y_1+y_2+y_3}{3},\displaystyle\frac{z_1+z_2+z_3}{3}\right),$$
所以$P_{1}\left(\displaystyle\frac{x_1+3\displaystyle\frac{x_2+x_3+x_4}{3}}{4},\displaystyle\frac{y_1+3\displaystyle\frac{y_2+y_3+y_4}{3}}{4},\displaystyle\frac{z_1+3\displaystyle\frac{z_2+z_3+z_4}{3}}{4}\right),$
即$P_{1}\left(\displaystyle\frac{x_1+x_2+x_3+x_4}{4},\displaystyle\frac{y_1+y_2+y_3+y_4}{4},\displaystyle\frac{z_1+z_2+z_3+z_4}{4}\right),$
同理可得$P_{2},P_{3},P_{4}$坐标,可知$P_{1},P_{2},P_{3},P_{4}$为同一点,故$A_{i}G_{i}$交于同一点$P$且点$P$到任一顶点的距离等于此点到对面重心的三倍. 

\item 证明:必要性:因为$\pi$上三点$p_{i}\left(x_i,y_i\right)_{i=1,2,3}$共线,故$\overrightarrow{p_1p_2}$平行于$\overrightarrow{p_1p_3}$,即$\displaystyle\frac{x_2-x_1}{x_3-x_1}=\displaystyle\frac{y_2-y_1}{y_3-y_1}$\\
即$x_1y_2+x_2y_3+x_3y_1-x_1y_3-x_3y_2-x_2y_1=\left|\begin{array}{ccc}x_1&y_1&1\\x_2&y_2&1\\x_3&y_3&1\end{array}\right|=0$.
充分性:由$\left|\begin{array}{ccc}x_1&y_1&1\\x_2&y_2&1\\x_3&y_3&1\end{array}\right|=x_1y_2+x_2y_3+x_3y_1-x_1y_3-x_3y_2-x_2y_1=0$整理得$$\displaystyle\frac{x_2-x_1}{x_3-x_1}=\displaystyle\frac{y_2-y_1}{y_3-y_1},$$
即$\overrightarrow{p_1p_2}$平行于$\overrightarrow{p_1p_3}$,所以$\pi$上三点$p_{i}\left(x_i,y_i\right)_{i=1,2,3}$共线. 
综上,$\pi$上三点$p_{i}\left(x_i,y_i\right)_{i=1,2,3}$共线当且仅当$\left|\begin{array}{ccc}x_1&y_1&1\\x_2&y_2&1\\x_3&y_3&1\end{array}\right|=0.$

\item 证明:建立仿射坐标系$\left\{\overrightarrow{A},\overrightarrow{AB},\overrightarrow{AC}\right\},$由$\overrightarrow{AP}=\lambda\overrightarrow{PB}=\lambda\left(\overrightarrow{AB}-\overrightarrow{AP}\right),$得$\overrightarrow{AP}=\displaystyle\frac{\lambda}{\lambda+1}\overrightarrow{AB},\overrightarrow{AP}=\left(\displaystyle\frac{\lambda}{\lambda+1},0\right)$;\\
$\overrightarrow{AR}=\displaystyle\frac{1}{1+\upsilon}\overrightarrow{AC},\overrightarrow{AR}=\left(0,\displaystyle\frac{1}{1+\upsilon}\right)$;\\
$\overrightarrow{AQ}=\displaystyle\frac{1}{1+\mu}\overrightarrow{AB}+\displaystyle\frac{\mu}{1+\mu}\overrightarrow{AC},\overrightarrow{AQ}=\left(\displaystyle\frac{1}{1+\mu},\displaystyle\frac{\mu}{1+\mu}\right)$;\\
由$P,Q,R$共线当且仅当$\left|\begin{array}{ccc}\displaystyle\frac{1}{1+\mu}&0&1\\ 0&\displaystyle\frac{1}{1+\upsilon}&1\\ \displaystyle\frac{1}{1+\mu}&\displaystyle\frac{\mu}{1+\mu}&1\end{array}\right|=0$,得$\lambda\mu\upsilon=-1$,证毕. \\
(注:事实上,此即平面几何上的梅涅劳斯定理)
\end{enumerate}

\subsection{数量积}
\begin{enumerate}
\item $\mathbf{a}\mathbf{b}+\mathbf{b}\mathbf{c}+\mathbf{c}\mathbf{a}=\displaystyle\displaystyle\frac{1}{2}\left[\left(\mathbf{a}+\mathbf{b}+\mathbf{c}\right)-\left(\mathbf{a}+\mathbf{b}+\mathbf{c}\right)\right]=-13$

\item $\left(3\mathbf{a}+2\mathbf{b}\right)\left(2\mathbf{a}-5\mathbf{b}\right)=6\left|\mathbf{a}\right|^2-0\left|\mathbf{b}\right|^2-11\mathbf{a}\mathbf{b}=14-33\sqrt{3}.$

\item 由题,得$$\left(\mathbf{a}+3\mathbf{b}\right)\left(7\mathbf{a}-5\mathbf{b}\right)=\left(\mathbf{a}-4\mathbf{b}\right)\left(7\mathbf{a}-2\mathbf{b}\right)=0$$解得:$\mathbf{a}\mathbf{b}=\displaystyle\frac{1}{2}\left|\mathbf{b}\right|^2$且$\left|\mathbf{a}\right|=\left|\mathbf{b}\right|$,知$\cos\angle\left(\mathbf{a},\mathbf{b}\right)=\displaystyle\frac{\mathbf{a}\mathbf{b}}{\left|\mathbf{a}\right|\left|\mathbf{b}\right|}=\displaystyle\frac{1}{2},$故$\angle\left(\mathbf{a},\mathbf{b}\right)=\displaystyle\frac{\pi}{3}$

\item \begin{enumerate}[(1)]
\item 错误:数量的概念不等同于向量概念;
\item 正确;
\item 错误:向量相等的必要条件是方向相同;
\item 错误:左边$=\left|\mathbf{a}\right|\left|\mathbf{b}\right|\cos^{2}\theta$,右边$=\left|\mathbf{a}\right|\left|\mathbf{b}\right|$;
\item 错误:向量相等的必要条件是方向相同;
\item 错误:左边$=\left|\mathbf{c}\right|\cdot\left|\mathbf{a}\right|\cdot\cos\angle\left(\mathbf{c},\mathbf{a}\right)\neq\left|\mathbf{c}\right|\cdot\left|\mathbf{b}\right|\cdot \cos\angle\left(\mathbf{c},\mathbf{b}\right)=$右边;
\end{enumerate}
\item 证明:左边$=\left(\mathbf{a}+\mathbf{b}\right)^2+\left(\mathbf{a}-\mathbf{b}\right)^2=2\mathbf{a}^2+2\mathbf{b}^2+2\mathbf{a}\mathbf{b}-2\mathbf{a}\mathbf{b}=$右边.\\
(注:几何含义为平行四边形两斜边的平方和等于四条边长的平方和)

\item \begin{enumerate}[(1)]
\item 证明:由向量乘法交换律得$$\left(\mathbf{a}\cdot\mathbf{b}\right)\left(\mathbf{a}\cdot\mathbf{c}\right)=\left(\mathbf{a}\cdot\mathbf{c}\right)\left(\mathbf{a}\cdot\mathbf{b}\right),$$故$\mathbf{a}\left[\left(\mathbf{a}\cdot\mathbf{b}\right)\cdot\mathbf{c}-\left(\mathbf{a}\cdot\mathbf{c}\right)\cdot\mathbf{b}\right]=0,$所以两向量垂直. \\
(注:$\left(\mathbf{a}\cdot\mathbf{c}\right)\cdot\mathbf{b}-\left(\mathbf{a}\cdot\mathbf{b}\right)\cdot\mathbf{c}=\mathbf{a}\times\mathbf{b}\times\mathbf{c}$不一定成立.)

\item 证明:因为$\mathbf{v_1},\mathbf{v_2}$不共线,取该平面任意向量$\mathbf{c}=\lambda\mathbf{v_1}+\mu\mathbf{v_2},$则$$\left(\mathbf{a}-\mathbf{b}\right)\mathbf{c}=\left(\mathbf{a}-\mathbf{b}\right)\left(\lambda\mathbf{v_1}+\mu\mathbf{v_2}\right)=\lambda\left(\mathbf{a}\mathbf{v_1}-\mathbf{b}\mathbf{v_1}\right)+\mu\left(\mathbf{a}\mathbf{v_2}-\mathbf{b}\mathbf{v_2}\right)=0$$故$\left(\mathbf{a}-\mathbf{b}\right)\perp\mathbf{c},$由$\mathbf{c}$的任意性得$\mathbf{a}-\mathbf{b}=\mathbf{0},$所以$\mathbf{a}=\mathbf{b}.$

\item 证明:假设$\mathbf{r}\neq\mathbf{0},$由题意,得$$\mathbf{r}\mathbf{a}-\mathbf{r}\mathbf{b}=0$$得$\mathbf{a}=\mathbf{b};$同理可得$\mathbf{a}=\mathbf{c},\mathbf{b}=\mathbf{c},$这与$\mathbf{a},\mathbf{b},\mathbf{c}$不共面矛盾,故$\mathbf{r}=\mathbf{0}.$
\end{enumerate}
\end{enumerate}

\subsection{向量积}
\begin{enumerate}
\item A.
\item A.
\item $\mathbf{a}\times\mathbf{b}\\
=\left(2\mathbf{m}-\mathbf{n}\right)\times\left(4\mathbf{m}-5\mathbf{n}\right)\\
=8\left(\mathbf{m}\times\mathbf{m}\right)-10\mathbf{m}\times\mathbf{n}-4\mathbf{n}\times\mathbf{m}+5\mathbf{n}\times\mathbf{n}\\
=-6\mathbf{m}\times\mathbf{n}$\\
得$\left|\mathbf{a}\times\mathbf{b}\right|=6\left|\mathbf{m}\times\mathbf{n}\right|=3\sqrt{2}.$

\item 因为$\mathbf{a}\times\mathbf{b}=\left(\left|\begin{array}{cc}3&-1\\-2&3\end{array}\right|,\left|\begin{array}{cc}-1&2\\3&1\end{array}\right|,\left|\begin{array}{cc}2&3\\1&-2\end{array}\right|\right)=\left(7,-7,-7\right).$
\begin{enumerate}[(1)]
\item 令$\mathbf{m}=\left(7,-7,-7\right),$则$$\mathbf{c}=\displaystyle\frac{\mathbf{m}}{\left|\mathbf{m}\right|}=\left(\displaystyle\frac{7}{7\sqrt{3}},\displaystyle-\frac{7}{7\sqrt{3}},-\displaystyle\frac{7}{7\sqrt{3}}\right)=\left(\displaystyle\frac{1}{\sqrt{3}},\displaystyle-\frac{1}{\sqrt{3}},-\displaystyle\frac{1}{\sqrt{3}}\right)$$
\item $\mathbf{c}=\lambda\left(\mathbf{a}\times\mathbf{b}\right)=\left(7\lambda,-7\lambda,-7\lambda\right)$\\
$\mathbf{c}\times\mathbf{d}=10$\\
所以$\lambda=\displaystyle\frac{5}{28},$\\
所以$\mathbf{c}=\left(\displaystyle\frac{5}{4},-\displaystyle\frac{5}{4},-\displaystyle\frac{5}{4}\right).$\end{enumerate}

\item 易证.

\item $\left(\mathbf{a}-\mathbf{d}\right)\times\left(\mathbf{b}-\mathbf{c}\right)\\
=\mathbf{a}\times\left(\mathbf{b}-\mathbf{c}\right)-\mathbf{d}\times\left(\mathbf{b}-\mathbf{c}\right)\\
=\mathbf{a}\times\mathbf{b}-\mathbf{a}\times\mathbf{c}-\mathbf{d}\times\mathbf{b}+\mathbf{d}\times\mathbf{c}\\
=\mathbf{c}\times\mathbf{d}-\mathbf{b}\times\mathbf{d}-\mathbf{d}\times\mathbf{b}+\mathbf{d}\times\mathbf{c}\\
=\mathbf{0}$\\
所以$\mathbf{a}-\mathbf{d}$与$\mathbf{b}-\mathbf{c}$共线. 
\end{enumerate}

\subsection{混合积与双重向量积}
\begin{enumerate}
\item D.\\
解:\begin{enumerate}[(A.)]
\item $\left|\mathbf{a}\right|\left|\mathbf{b}\right|\cos<\mathbf{a},\mathbf{b}>=\left|\mathbf{a}\right|\left|\mathbf{c}\right|\cos<\mathbf{a},\mathbf{c}>\left(\left|\mathbf{a}\right|\neq0\right).$
\item 取$\mathbf{a}=\mathbf{0}$或$\mathbf{b}=\mathbf{0}.$
\item 取$\mathbf{a}=\mathbf{0}.$
\item 证明:原式左右两边同乘以向量$\mathbf{c},$得$$\mathbf{a}\times\mathbf{b}\cdot\mathbf{c}+\mathbf{b}\times\mathbf{c}\cdot\mathbf{c}+\mathbf{c}\times\mathbf{a}\cdot\mathbf{c}=0$$由定理1.6与命题1.6.1得$$\left(\mathbf{a},\mathbf{b},\mathbf{c}\right)=0$$由推论1.6.1,命题得证. \end{enumerate}

\item C.\\
解:$\mathbf{a}\left[\left(\mathbf{c}\cdot\mathbf{b}\right)\mathbf{b}-\left(\mathbf{a}\cdot\mathbf{b}\right)\mathbf{c}\right]=\left(\mathbf{a}\cdot\mathbf{b}\right)\left(\mathbf{c}\cdot\mathbf{d}\right)-\left(\mathbf{a}\cdot\mathbf{c}\right)\left(\mathbf{a}\cdot\mathbf{c}\right)=0$,又$\mathbf{a},\mathbf{b}\mathbf{c}\neq\mathbf{0}$,得证.\\
(注:定理1.6.2不一定成立,一位内向量叉乘只有在$\mathbb{R}^3$情况下才成立.)

\item 解:与例1.6.1同理,$V=\displaystyle\frac{59}{6}$.

\item \begin{enumerate}[(1)]
\item 同理,$A,B,C,D$四点共面. 
\item $V=\displaystyle\frac{1}{6}\left|\left(\overrightarrow{AB},\overrightarrow{AC},\overrightarrow{AD}\right)\right|=\displaystyle\frac{58}{3},$\\
$h_D=\displaystyle\frac{6V}{\left|\overrightarrow{AB}\times\overrightarrow{AC}\right|}=\displaystyle\frac{6V}{\left|\begin{array}{ccc}1&1&1\\2&-2&-3\\4&0&6\end{array}\right|}=\displaystyle\frac{29}{7}$
\end{enumerate}

\item $\displaystyle\frac{8}{25},\displaystyle\frac{5}{2}$

\item \begin{enumerate}[(1)]
\item 证明:综合运用命题1.6.1可证得.
\item 证明:左边$=\left(\mathbf{a},\mathbf{b}+\mathbf{c},\mathbf{c}+\mathbf{a}\right)+\left(\mathbf{b},\mathbf{b}+\mathbf{c},\mathbf{c}+\mathbf{a}\right)\\
=\left(\mathbf{a},\mathbf{b}+\mathbf{c},\mathbf{c}\right)+\left(\mathbf{a},\mathbf{b}+\mathbf{c},\mathbf{a}\right)+\cdots\\
=\cdots\\
=2\left(\mathbf{a},\mathbf{b},\mathbf{c}\right)=$右边. 
\item 证明:同(2)理,展开右边即得.\\
(注:类比$\left(\mathbf{a}-\mathbf{d}\right)\left(\mathbf{b}-\mathbf{d}\right)\left(\mathbf{c}-\mathbf{d}\right)=\mathbf{a}\mathbf{b}\mathbf{c}-\mathbf{a}\mathbf{b}\mathbf{d}-\mathbf{d}\mathbf{b}\mathbf{c}-\mathbf{a}\mathbf{d}\mathbf{c}+0\left(\mathbf{a}\mathbf{d}\mathbf{d}+\mathbf{b}\mathbf{d}\mathbf{d}+\mathbf{c}\mathbf{d}\mathbf{d}-\mathbf{d}\mathbf{d}\mathbf{d}\right)$)
\item 证明:左边$=\left(\mathbf{a}+\mathbf{b}\right)\left(\mathbf{a}\times\mathbf{c}+\mathbf{b}\times\mathbf{c}\right)=\mathbf{a}\cdot\left(\mathbf{a}\times\mathbf{c}\right)+\mathbf{a}\cdot\left(\mathbf{b}\times\mathbf{c}\right)+\mathbf{b}\cdot\left(\mathbf{b}\times\mathbf{c}\right)+\mathbf{b}\cdot\left(\mathbf{a}\times\mathbf{c}\right)=$右边.
\item 证明:设$\mathbf{d}=\lambda\mathbf{a}+\mu\mathbf{b}+\upsilon\mathbf{c},$\\则$\left(\mathbf{a},\mathbf{b},\mathbf{c}\times\mathbf{d}\right)=\left(\mathbf{a}\times\mathbf{b}\right)\left[\mathbf{c}\times\left(\lambda\mathbf{a}+\mu\mathbf{b}+\upsilon\mathbf{c}\right)\right]\\
=\left(\mathbf{a}\times\mathbf{b}\right)\left[\mathbf{c}\times\left(\lambda\mathbf{a}+\mu\mathbf{b}\right)\right]\\
=\mathbf{a}\times\mathbf{b}\left(\lambda\mathbf{c}\times\mathbf{a}+\mu\mathbf{c}\times\mathbf{b}\right)\\
=\lambda\left(\mathbf{a}\times\mathbf{b}\right)\left(\mathbf{c}\times\mathbf{a}\right)+\mu\left(\mathbf{a}\times\mathbf{b}\right)\left(\mathbf{c}\times\mathbf{b}\right)\textcircled{1}\\$
同理展开其余两式,得\\
$\left(\mathbf{b},\mathbf{c},\mathbf{a}\times\mathbf{d}\right)=\mu\left(\mathbf{b}\times\mathbf{c}\right)\left(\mathbf{a}\times\mathbf{b}\right)+\upsilon\left(\mathbf{b}\times\mathbf{c}\right)\left(\mathbf{a}\times\mathbf{c}\right)\textcircled{2}\\$
$\left(\mathbf{c},\mathbf{a},\mathbf{b}\times\mathbf{d}\right)=\lambda\left(\mathbf{c}\times\mathbf{a}\right)\left(\mathbf{b}\times\mathbf{a}\right)+\upsilon\left(\mathbf{c}\times\mathbf{a}\right)\left(\mathbf{b}\times\mathbf{c}\right)$\textcircled{3}\\
$\textcircled{1}+\textcircled{2}+\textcircled{3}$,整理得\\
左边$=\lambda\left[\left(\mathbf{a}\times\mathbf{b}\right)\left(\mathbf{c}\times\mathbf{a}\right)\\
+\left(\mathbf{c}\times\mathbf{a}\right)\left(\mathbf{b}\times\mathbf{a}\right)\right]\\
+\mu\left[\left(\mathbf{a}\times\mathbf{b}\right)\left(\mathbf{c}\times\mathbf{b}\right)+\left(\mathbf{b}\times\mathbf{c}\right)\left(\mathbf{a}\times\mathbf{b}\right)\right]\\
+\upsilon\left[\left(\mathbf{b}\times\mathbf{c}\right)\left(\mathbf{a}\times\mathbf{c}\right)+\left(\mathbf{c}\times\mathbf{a}\right)\left(\mathbf{b}\times\mathbf{c}\right)\right]\\
=\lambda\cdot0+\mu\cdot0+\upsilon\cdot0\\
=0=$右边.\\
等式得证. 
\end{enumerate}

\item 证明:显然$\mathbf{a},\mathbf{b},\mathbf{c}\perp\mathbf{n},$则$\mathbf{a},\mathbf{b},\mathbf{c}$共面.否则:\\
若$\mathbf{n}=\mathbf{0}$,则$\mathbf{a}=\mathbf{b}=\mathbf{c}=\mathbf{0},$仍成立;\\
若$\mathbf{n}\neq\mathbf{0}$,$\mathbf{a},\mathbf{b},\mathbf{c}$中至少有两个向量共线,则仍成立;\\
若$\mathbf{n}\neq\mathbf{0}$,$\mathbf{a},\mathbf{b},\mathbf{c}$胡不共线,则$\mathbf{n}$为$\mathbf{a},\mathbf{b}$所确定的平面的法向量,$\mathbf{n}\cdot\mathbf{c}\neq0$,这与题设相悖.\\
故成立. 
\end{enumerate}

\section{平面与直线}
\subsection{平面方程}
\begin{enumerate}
\item \begin{enumerate}[(1)]
\item 取$\mathbb{Z}$轴上两点和题设点$\left(0,0,0\right),\left(0,0,1\right),\left(3,1,-2\right).$\\
所求方程为$\left|\begin{array}{ccc}x&y&z\\0&0&1\\3&1&-2\end{array}\right|=x-3y=0$
\item 由平面点法式方程,不妨设所求平面方程为$3x-2y+5=D\left(D\neq0\right).$\\
代入点$\left(-1,-5,4\right)$得$3x-2y-7=0.$
\item 不妨设所求平面法向量为$\mathbf{n}=\left(a,b,c\right).$\\
则$\mathbf{n}\cdot\overrightarrow{M_1M_2}=\mathbf{n}\cdot\left(1,-8,3\right)=0.$\\
即$\left\{\begin{array}{c}a+6b+c=0\\a-8b+3c=0\end{array}\right.$,取一组解$\mathbf{n}=\left(13,-1,-7\right).$\\
同(2)理可得$13x-y-7z=37.$\end{enumerate}

\item $\overrightarrow{AB}=\left(-4,5,-1\right),\overrightarrow{CD}=\left(-1,0,2\right).$\\
由题得平面的法向量为$\mathbf{n_1}=\overrightarrow{AB}\times\overrightarrow{CD}=\left(11,7,5\right).$\\
得此平面方程为$11\left(x-4\right)+7\left(y-0\right)+5\left(z-6\right)=0$\\
即$11x+7y+5z=74.$\\
$\overrightarrow{AB}=\left(-4,5,-1\right),\overrightarrow{BC}=\left(-4,-6,2\right).$\\
由题得平面$ABC$的法向量为$\mathbf{n_2}=\overrightarrow{AB}\times\overrightarrow{BC}=\left(4,3,1\right).$\\
所以平面的法向量$\mathbf{n_3}=\mathbf{n_2}\times\overrightarrow{AB}=\left(-8,0,32\right).$
得此平面方程为$-8\left(x-5\right)+32\left(z-3\right)=0.$\\
即$x-4z+7=0.$

\item $x+2y-z+4=0\Leftrightarrow x+2y-z=-4\Leftrightarrow \displaystyle\frac{x}{-4}+\displaystyle\frac{y}{-2}+\displaystyle\frac{z}{4}=1$\\
由此知平面过坐标轴上$A\left(-4,0,0\right),B\left(0,-2,0\right),C\left(0,0,4\right)$\\
知道$\overrightarrow{AB}=\left(4,-2,0\right),\overrightarrow{AC}=\left(4,0,4\right).$\\
得参数方程$\left\{\begin{array}{l}x=-4+2u+v\\y=-u\\z=v\end{array}\right.$\end{enumerate}

\subsection{直线方程}
\begin{enumerate}
\item \begin{enumerate}[(1)]
\item 取直线的法向量$\mathbf{v}$使与已知平面的法向量$\mathbf{n_1}$平行,令$$\mathbf{v}=\lambda\mathbf{n_1}=\left(6,-3,-5\right)$$由$M\left(2,-3,-5\right)$得点向式直线标准方程$\displaystyle\frac{x-2}{6}=\displaystyle\frac{y+3}{-3}=\displaystyle\frac{z+5}{-5}.$
\item 取所求直线的方向向量$\mathbf{v}=\left(x_0,y_0,z_0\right)$与已知两直线的方向向量$\mathbf{n_1}=\left(1,1,-1\right),\mathbf{n_2}=\left(1,-1,0\right)$垂直.\\
即$\mathbf{v}\cdot\mathbf{n_1}=\mathbf{v}\cdot\mathbf{n_2}=0,$取$\mathbf{v}$的一组解为$\left(1,1,2\right)$,又$M\left(1,0,-2\right)$\\
得点向式直线标准方程为$x-1=y=\displaystyle\frac{z+2}{2}.$
\item 解:设所求直线的方向向量为$\mathbf{v}=\left(x_0,y_0,z_0\right)$,由题得$$\mathbf{v}=\left(\cos\alpha,\cos\beta,\cos\gamma\right)=\left(\displaystyle\frac{1}{2},\displaystyle\frac{\sqrt{2}}{2},\displaystyle-\frac{1}{2}\right)$$\\
又$M\left(1,-5,3\right)$得直线点向式标准方程为$x-1=\displaystyle\frac{y+5}{\sqrt{2}}=\displaystyle\frac{y-3}{-1}.$
\item 解:设所求直线的方向向量为$\mathbf{v_0}=\left(x_0,y_0,z_0\right),$\\
已知平面的法向量为$\mathbf{n}=\left(3,-1,2\right)$\\
已知直线的方面向量为$\mathbf{v}=\left(4,-2,1\right)$\\
已知直线上的一个点$Q\left(1,3,0\right),$由题得$\mathbf{v_0}\cdot\mathbf{n}=0,$且$\left(\overrightarrow{PQ},\mathbf{v_0},\mathbf{v}\right)=0$(注:本书第41页命题2.3.2(2))\\
即$\left\{\begin{array}{l}3x_0-y_0+2z_0=0\\\left|\begin{array}{ccc}x_0&y_0&z_0\\0&3&2\\4&-2&1\end{array}\right|=0\end{array}\right.$\\
得直线点向式标准方程为$\displaystyle\frac{x-1}{-4}=\displaystyle\frac{y}{50}=\displaystyle\frac{z+2}{31}$
\end{enumerate}

\item \begin{enumerate}[(1)]
\item 解:令$y=0$得一点$M\left(-5,0,-9\right)\in l,$\\
已知平面的法向量为$\mathbf{n_1}=\left(2,1,-1\right),\mathbf{n_2}=\left(3,-1,-2\right)$\\
取所求直线的方向向量$\mathbf{v}=\mathbf{n_1}\times\mathbf{n_2}=\left(-3,1,-5\right),$\\
则该直线点向式方程为$\displaystyle\frac{x+5}{-3}=y=\displaystyle\frac{z+9}{-5}.$
\item 解:从一般方程中消去$z$,得$4y=3x$,消去$x$,得$4y=-3z+18$,于是得$$\displaystyle\frac{x}{4}=\displaystyle\frac{y}{3}=\displaystyle\frac{z-6}{-4}.$$
\end{enumerate}

\item \begin{enumerate}[(1)]
\item 解:设以$\left\{\begin{array}{l}2x-7y+4z-3=0\\3x-5y+4z+11=0\end{array}\right.$为轴的平面束是$$\lambda\left(2x-7y+4z-3\right)+\mu\left(3x-5y+4z+11\right)=0$$代入点$\left(-2,1,3\right)$得$\mu/\lambda=-1/6,$\\
故所求平面方程为$\displaystyle\frac{3}{2}x-\displaystyle\frac{37}{6}y+\displaystyle\frac{10}{3}z-\displaystyle\frac{19}{6}=0.$
\item 解:同理,设平面束方程为$$\lambda\left(2x-7y+4z-3\right)+\mu\left(3x-5y+4z+11\right)=0$$取其法向量$\mathbf{n_1}=\left(2\lambda+3\mu,-7\lambda-5\mu,4\lambda+4\mu\right)$与已知平面法向量$\mathbf{n_2}=\left(1,1,1\right)$垂直,即$\mathbf{n_1}\cdot\mathbf{n_2}=0,$解得$\mu/\lambda=-1/2,$\\
故所求平面方程为$\displaystyle\frac{3}{2}x-\displaystyle\frac{3}{2}z+2=0.$
\item \begin{enumerate}[(a)]
\item 解法一;任取过直线$\displaystyle\frac{x-1}{2}=\displaystyle\frac{y+2}{-3}=\displaystyle\frac{z-2}{x}$的平面方程$$\pi_1:2\displaystyle\frac{x-1}{2}-\displaystyle\frac{y+2}{-3}-\displaystyle\frac{z-2}{2}=0$$$$\pi_2:\displaystyle\frac{x-2}{2}+\displaystyle\frac{y+2}{-3}-2\displaystyle\frac{z-2}{2}=0$$化简得$$\pi_1:6x+2y-3z+4=0$$$$\pi_2:3x-2y-6z+2=0$$同理,得所求平面方程为$x-8y-13z+9=0.$
\item 解法二:已知平面的法向量$\mathbf{n_1}=\left(3,2,1\right)$\\
取已知直线上一点$\left(1,-2,-2\right)$及其方向向量$\mathbf{v}=\left(2,-3,2\right)$\\
则所求平面的法向量$\mathbf{n_2}=\mathbf{n_1}\times\mathbf{v}=\left(1,-12,13\right)$\\
又所求平面过点$\left(1,-2,2\right)$得所求平面方程为$$x-8y-13z+9=0.$$
\end{enumerate}
\item 解:设该平面为$Ax+By+Cz+D=0$,\\
因为该平面与直线$\displaystyle\frac{x-2}{-1}=\displaystyle\frac{y+4}{3}=\displaystyle\frac{z+1}{1}$垂直,有$\mathbf{n}$平行$\mathbf{v}$.\\
所以$\mathbf{n}=\left(-1,3,1\right)$,则有$-x+3y+z+D=0.$\\
又因为该平面过点$\left(4,-1,2\right)$,所以得$D=5$,\\
综上,该平面方程为$-x+3y+z+5=0.$
\item 解:易得直线$\left\{\begin{array}{l}2x-y-z-3=0\\x+2y-z-5=0\end{array}\right.$的方向向量$$\mathbf{v_0}=\left(2,-1,-1\right)\times\left(1,2,-1\right)=\left(3,1,5\right)$$记直线$\displaystyle\frac{x-2}{1}=\displaystyle\frac{y+3}{-5}=\displaystyle\frac{z+1}{-1}$的方向向量为$\mathbf{v}$,得$\mathbf{v}=\left(1,-5,-1\right)$,\\
设所求平面的法向量为$\mathbf{n}$,则$\mathbf{n}\cdot\mathbf{v_0}=\mathbf{n}\cdot\mathbf{v}=0.$\\
取$\mathbf{n}$的一组非零解$\left(3,1,-2\right)$\\
代入直线$\displaystyle\frac{x-2}{1}=\displaystyle\frac{y+3}{-5}=\displaystyle\frac{z+1}{-1}$上一点$\left(2,-3,-1\right)$得\\所求平面方程为$$3x+y-2z-5=0$$.
\end{enumerate}
\end{enumerate}

\subsection{线、面的位置关系}
\begin{enumerate}
\item \begin{enumerate}[(1)]
\item 解:取$M_1\left(-1,1,2\right)\in\displaystyle\frac{x+1}{3}=\displaystyle\frac{y-1}{3}=\displaystyle\frac{z-2}{1},$\\
取$M_2\left(0,6,-5\right)\in\displaystyle\frac{x}{-1}=\displaystyle\frac{y-6}{2}=\displaystyle\frac{z+5}{3},$\\
由两已知直线的方向向量分别为$\mathbf{v_1}=\left(3,3,1\right),\mathbf{v_2}=\left(-1,2,3\right)$,\\根据命题2.3.2,因为$\left(\overrightarrow{M_1M_2},\mathbf{v_1},\mathbf{v_2}\right)=106\neq0$知两平面异面.

\item 解:\\$\left\{\begin{array}{l}x+y+z=0\\y+z+1=0\end{array}\right.\Leftrightarrow\displaystyle\frac{x-1}{0}=\displaystyle\frac{y-1}{-1}=\displaystyle\frac{z+2}{1},$\\
$\left\{\begin{array}{l}x+z+1=0\\x+y+1=0\end{array}\right.\Leftrightarrow\displaystyle\frac{x+1}{-1}=\displaystyle\frac{y}{1}=\displaystyle\frac{z}{1},$\\
根据命题2.3.2,计算三向量的混合积为$3$,知两平面异面. 
\end{enumerate}

\item \begin{enumerate}[(1)]
\item 解:$\mathbb{X}$轴所在直线方程为$y=z=0$,则联立方程$$\left\{\begin{array}{l}A_1x+B_1y+C_1z+D_1=0\\A_2x+B_2y+C_2z+D_2=0\\0\cdot x+0\cdot y+z+0=0\\0\cdot x+y+0\cdot z+0=0\end{array}\right.$$由例2.3.2得所求条件为$$\left|\begin{array}{cccc}A_1&B_1&C_1&D_1\\A_2&B_2&C_2&D_2\\0&0&1&0\\0&1&0&0\end{array}\right|=0$$即$A_1D_2=A_2D_1$
\item 设所给直线的方向向量$\mathbf{v}=\mathbf{n_1}\times\mathbf{n_2}=\left(A_1,B_1,C_1\right)\times\left(A_2,B_2,C_2\right)=\left(\lambda,0,0\right),\lambda\in\mathbb{R}\backslash\{0\}$\\
得$\left|\begin{array}{cc}A_1&B_1\\A_2&B_2\end{array}\right|=\left|\begin{array}{cc}A_1&C_1\\A_2&C_2\end{array}\right|=0,$且$\left|\begin{array}{cc}B_1&C_1\\B_2&C_2\end{array}\right|\neq0$\\
令直线不与$\mathbb{X}$轴重合,只需令$\left(0,0,0\right)$不满足直线方程,即$D_1=D_2=0$无解.\\
综上,所求条件为$\left\{\begin{array}{l}\left|\begin{array}{cc}A_1&B_1\\A_2&B_2\end{array}\right|=\left|\begin{array}{cc}A_1&C_1\\A_2&C_2\end{array}\right|=0\\\left|\begin{array}{cc}B_1&C_1\\B_2&C_2\end{array}\right|\neq0\\D_2\neq0\end{array}\right.$或$\left\{\begin{array}{l}\left|\begin{array}{cc}A_1&B_1\\A_2&B_2\end{array}\right|=\left|\begin{array}{cc}A_1&C_1\\A_2&C_2\end{array}\right|=0\\\left|\begin{array}{cc}B_1&C_1\\B_2&C_2\end{array}\right|\neq0\\D_1\neq0\end{array}\right.$
\item 同理可得所求条件为$\left|\begin{array}{cc}A_1&B_1\\A_2&B_2\end{array}\right|=\left|\begin{array}{cc}A_1&C_1\\A_2&C_2\end{array}\right|=D_1=D_2=0$\\
(注:若$\left|\begin{array}{cc}A_1&B_1\\A_2&B_2\end{array}\right|=\left|\begin{array}{cc}A_1&C_1\\A_2&C_2\end{array}\right|=\left|\begin{array}{cc}B_1&C_1\\B_2&C_2\end{array}\right|=0$则$\left\{\begin{array}{l}A_1x+B_1y+C_1z+D_1=0\\A_2x+B_2y+C_2z+D_2=0\end{array}\right.$不构成直线,两平面平行或重合.)
\end{enumerate}

\item \begin{enumerate}[(1)]
\item 解:设所给直线方向向量为$\mathbf{v}=\left(3,-2,7\right),$所给平面法向量为$\mathbf{n}=\left(4,-3,7\right),$\\
$\mathbf{v}\cdot\mathbf{n}\neq0$,且显然$\mathbf{v}$不平行于$\mathbf{n}$,则直线与平面相交. 
\item 解:由题,得$\left\{\begin{array}{l}5x-3y+2z-5=0\\2x-y-z-1=0\\4x-3y+7z-7=0\end{array}\right.$,由克拉默法则$D=\left|\begin{array}{ccc}5&-3&2\\2&-1&-1\\4&-3&7\end{array}\right|=0$知直线在平面上. 
\item 已知直线方向向量为$\mathbf{v}=\left(1,-2,9\right)$,\\
已知平面法向量为$\mathbf{n}=\left(3,-4,7\right)$\\
同(1)中理可得直线与平面交于一点. 
\end{enumerate}

\item 解:联立方程$\left\{\begin{array}{l}A_1x+B_1y+C_1z=0\\A_2x+B_2y+C_2z=0\\\left(A_1+A_2\right)x+\left(B_1+B_2\right)y+\left(C_1+C_2\right)z=0\end{array}\right.$\\
由克拉默法则$D=\left|\begin{array}{ccc}A_1&B_1&C_1\\A_2&B_2&C_2\\A_1+A_2&B_1+B_2&C_1+C_2\end{array}\right|=0$是显然的,\\
则在直线$\left\{\begin{array}{l}A_1x+B_1y+C_1z=0\\A_2x+B_2y+C_2z=0\end{array}\right.$上的点必在平面$$\left(A_1+A_2\right)x+\left(B_1+B_2\right)y+\left(C_1+C_2\right)z=0$$上,知直线在平面上. 

\item \begin{enumerate}[(1)]
\item 解:令直线的方向向量$\mathbf{v}$与所给平面法向量$\mathbf{n}$垂直,得$$\mathbf{v}\cdot\mathbf{n}=\left(4,3,1\right)\cdot\left(k,3,-5\right)=0$$得$k=-1$,经检验,直线不在平面内,故$k=-1$满足题意.
\item 解:令所给直线的方向向量$\mathbf{v}=\left(2,-4,3\right)$与所给平面法向量$\mathbf{n}=\left(k,m,6\right)$平行,即$$\left(2,-4,3\right)=\lambda\left(k,m,6\right)$$解得$k=4,m=8.$
\end{enumerate}

\item 解:设所给直线为$l_4$,其方向向量为$\mathbf{v}=\left(8,7,1\right)$,则$l_4$与$l_3$所确定的平面的法向量$\mathbf{n_1}=\mathbf{v_1}\times\mathbf{v_2}=\left(4,-6,10\right)$代入$l_2$上$\left(-13,5,0\right)$得$l_4$与$l_2$所确定的平面$$x-y+z-17=0$$知所求直线为$\left\{\begin{array}{l}2x-3y+5z+41=0\\x-y+z-17=0\end{array}\right.$.

\item 解:与例2.3.3同理,所求直线为$\left\{\begin{array}{l}2x+4y-3z-11=0\\13x-9y+2z-50=0\end{array}\right.$

\item 解:由题设可知:设未知直线的方向向量为$\mathbf{v}=\left(X,Y,Z\right)$,$l'$的方向向量为$\mathbf{v'}=\left(1,5,3\right)$,平面法向量为$\mathbf{n}=\left(2,1,-3\right)$,所以$\mathbf{v}\cdot\mathbf{v_1}=\mathbf{v}\cdot\mathbf{n}=0$,即$$\left\{\begin{array}{l}X+5Y+3Z=0\\2X+Y-3Z=0\end{array}\right.$$解得$X=-2Y,Z=-Y$,所以$X:Y:Z=-2:1:-1$,\\
又因为未知直线过$l'$与平面\uppercase\expandafter{\romannumeral2}的交点$P$,所以$$\left\{\begin{array}{l}2X+Y-3Z+1=0\\\displaystyle\frac{x}{1}=\displaystyle\frac{y+5}{5}=\displaystyle\frac{z+2}{3}=0\end{array}\right.$$所以$P\left(1,0,1\right)$,所以直线方程为$\displaystyle\frac{x-1}{-2}=\displaystyle\frac{y}{1}=\displaystyle\frac{z-1}{-1}$.

\item 解:设所求直线的方向向量为$\mathbf{v}$,所求平面的法向量为$\mathbf{n}$,所给直线的方向向量为$\mathbf{v_0}=\left(1,0,0\right)$,则$\mathbf{v}\cdot\mathbf{v_0}=\mathbf{v}\cdot\mathbf{n}=0$得$\mathbf{v}=\mathbf{v_0}\times\mathbf{n}=\left(0,-1,1\right)$取所给直线与平面的交点$\left(1,1,-1\right)$代入,得所求直线方程为$\displaystyle\frac{x-1}{-2}=\displaystyle\frac{y-1}{-1}=\displaystyle\frac{z+1}{1}$.

\item 解:化$l_0$方程为标准方程$\displaystyle\frac{x}{2}=\displaystyle\frac{y}{4}=\displaystyle\frac{z+\frac{1}{2}}{5}$,设$M_1$在$l_0$上的投影点为$M_0\left(2t,4t,\displaystyle\frac{1}{2}+5t\right)$,\\
则$\overrightarrow{M_1M_0}\cdot\left(2,4,5\right)=0$得$M_0\left(3,6,8\right)$.由中点坐标公式得$M_2\left(2,15,6\right)$,易得所求直线方程为$$\displaystyle\frac{x-2}{2}=\displaystyle\frac{y-15}{4}=\displaystyle\frac{z-6}{5}$$

\item 证明:联立方程得$$\left\{\begin{array}{l}cy+bz-bc=0\\x=0\\ax-az-ac=0\\y=0\end{array}\right.$$其判别式$$\left|\begin{array}{cccc}0&c&b&-bc\\1&0&0&0\\a&0&-a&-ac\\0&1&0&0\end{array}\right|=2abc\neq0,$$由例2.3.2,$l_1,l_2$不交于一点,则$l_1,l_2$异面或平行,而毫无疑问,$l_1,l_2$不平行,故得证.
\end{enumerate}

\subsection{点、线、面之间的距离}
\begin{enumerate}

\item (1)$d=\displaystyle\frac{20}{11}\sqrt{2}$,(2)$d=\sqrt{6}$.

\item 证明:由点到平面距离公式$p=1-\displaystyle\frac{11}{\sqrt{\left(\displaystyle\frac{1}{a}\right)^2+\left(\displaystyle\frac{1}{b}\right)^2+\left(\displaystyle\frac{1}{c}\right)^2}}$整理即得所求证式.

\item 解:显然平面$x+1=0$不满足条件,则设过直线$\displaystyle\frac{x+1}{0}=\displaystyle\frac{y+\frac{2}{3}}{2}=\displaystyle\frac{z}{-3}$的平面为$\left(x+1\right)+m\left(9y+2z+2\right)=0$,即$x+3my+2mz+2m+1=0$.\\
由点到平面距离公式得$$3^2=\displaystyle\frac{\left(4+3m+4m+2m+1\right)^2}{1^2+\left(3m\right)^2+\left(2m\right)^2}$$解得$m=d=\displaystyle-\frac{1}{6}$或$d=\displaystyle\frac{8}{3}$所以所求平面方程为$3x+24y+16z+19=0$或$3x-y-z+2=0$.

\item (1)$d=0$,(2)$d=\displaystyle\frac{15}{41}\sqrt{41}$,(3)$d=0$.

\item 解:\begin{enumerate}[(1)]
\item 证明:由定理2.4.2得公垂线段的长$d=\displaystyle\frac{3}{112}\sqrt{122}$,所以两直线异面,同时其公垂线方程为$$\left\{\begin{array}{l}\left|\begin{array}{ccc}x-1&y&z\\1&-3&3\\3&8&7\end{array}\right|=0\\
\left|\begin{array}{ccc}x&y&z\\2&1&-2\\1&8&7\end{array}\right|=0\end{array}\right.$$即$\left\{\begin{array}{l}-45x+2y+17z+45=0\\23x-20y+13z=0\end{array}\right.$
\item 同理可得其公垂线段的长$d=\sqrt{54}$,所以两直线异面,同时其公垂线方程为
$$\left\{\begin{array}{l}\left|\begin{array}{ccc}x+2&y&z-2\\1&0&-1\\5&-2&5\end{array}\right|=0\\
\left|\begin{array}{ccc}x-3&y+2&z-7\\1&5&1\\5&-2&5\end{array}\right|=0\end{array}\right.$$即$\left\{\begin{array}{l}x+5y+z=0\\x-y+4=0\end{array}\right.$
\item 同理可得其公垂线段的长$d=\displaystyle\frac{1}{6}$,所以两直线异面,同时其公垂线方程为$$\left\{\begin{array}{l}x+y+4z-1=0\\x-2y-2z+3=0\end{array}\right.$$
\end{enumerate}

\item 解:\begin{enumerate}[(1)]
\item 整理方程得$m\left(x+y\right)+n\left(-z-1\right)=0$\\
$m,n$作为变量,若$x,y$满足$\left\{\begin{array}{l}x+y=0\\-z-1=0\end{array}\right.$,则原方程恒成立.\\
知平面\uppercase\expandafter{\romannumeral2}恒过定直线$l_1:\left\{\begin{array}{l}x+y=0\\-z-1=0\end{array}\right.$,点$M_1\left(0,0,-1\right)$显然在直线$l_1$上.
\item $l_1:x=-y=\displaystyle\frac{z+1}{0}$,$l_1,l_2$的判别式$\Delta=\left|\begin{array}{ccc}-1&-1&-2\\1&-1&0\\1&1&0\end{array}\right|=-1\neq0$知$l_1,l_2$异面.
\item 由定理2.4.2得$l_1,l_2$间的距离$d=2$,其公垂线方程为
$$\left\{\begin{array}{l}\left|\begin{array}{ccc}x&y&z+1\\1&-1&0\\0&0&2\end{array}\right|=0\\
\left|\begin{array}{ccc}x-1&y-1&z-1\\1&1&0\\0&0&2\end{array}\right|=0\end{array}\right.$$
即$\left\{\begin{array}{l}x+y=0\\x-y=0\end{array}\right.$
\end{enumerate}

\item 解:过点$M_1$且与平面\uppercase\expandafter{\romannumeral2}平行的平面\uppercase\expandafter{\romannumeral3}为$3x-2y+z-6=0$,\\
过点$M_2$且与平面\uppercase\expandafter{\romannumeral2}平行的平面\uppercase\expandafter{\romannumeral4}为$3x-2y+z-13=0$,\\
由$-6<-4,-13<-4$,得$M_1,M_2$在平面\uppercase\expandafter{\romannumeral2}的同侧\\
(注:将方程$f\left(x,y,z\right)=0$的图像沿$\mathbb{Z}$轴正方向平移$a$单位长度(若$a<0$,则沿反方向平移$\left|a\right|$单位长),则得到的图像方程为$f\left(x,y,z-a\right)=0$,故若平面在平面\uppercase\expandafter{\romannumeral2}的$\mathbb{Z}$轴方向上方,则其标准方程常数项比\uppercase\expandafter{\romannumeral2}小,反之同理)

\item \begin{enumerate}[(1)]
\item 解得下列平面:\\过$M$平行$\pi_1:3x-y+2z-9=0$\\过$M$平行$\pi_2:x-2y-z-3=0$\\过$N$平行$\pi_1:3x-y+2z+5=0$\\过$N$平行$\pi_2:x-2y-z=0$\\
则由第7题注,得$M$在$\pi_1$上,在$\pi_2$下;$N$在$\pi_1$下,在$\pi_2$下,知$N,M$两点在相邻的二面角内. 
\item 同理可得,$N,M$两点在对顶的二面角内.
\end{enumerate}

\item 证明:假设有两条公垂线,则它们都与异面直线相交,所以公垂线确定一个平面$A$,所以四个交点共面,又因为每条异面直线都有四个点在平面$A$上,所以异面直线都在平面$A$上,所以两直线共面,与题设矛盾,假设不成立. 故原命题成立. 
\end{enumerate}

\subsection{线、面间的夹角}
略.

\subsection{平面束}
\begin{enumerate}
\item 证明:$l$的方程为$\displaystyle\frac{x-2}{-1}=\displaystyle\frac{y-1}{4}=\displaystyle\frac{z}{3}$,即$$\left\{\begin{array}{l}\displaystyle\frac{x-2}{-1}=\displaystyle\frac{z}{3}\\\displaystyle\frac{x-2}{-1}=\displaystyle\frac{y-1}{4}\end{array}\right.$$代入平面方程中则易知恒满足平面方程,知$l$在平面\uppercase\expandafter{\romannumeral2}上. 

\item 解:\begin{enumerate}[(1)]
\item 设所求平面为$\mu\left(4x-y+3z-1\right)+\lambda\left(x+5y-z+2\right)=0$,即$\left(4\mu+\lambda\right)x+\left(-\mu+5\lambda\right)y+\left(3\mu-\lambda\right)z+\left(-\mu+2\lambda\right)=0$\\
令$\left(4\mu+\lambda,-\mu+5\lambda,3\mu-\lambda,\right)\cdot\left(0,1,0\right)=0$,得$5\lambda=\mu$,得所求方程为$21x+14y-3=0$
\item 同理,令$\left(4\mu+\lambda,-\mu+5\lambda,3\mu-\lambda\right)\cdot\left(2,-1,5\right)=0$\\
得平面方程为$7x+14y+5=0.$
\end{enumerate}

\item 解:\begin{enumerate}[(1)]
\item 设所求平面为$\lambda\left(\displaystyle\frac{x+1}{2}-\displaystyle\frac{y}{-1}\right)+\lambda\left(\displaystyle\frac{y}{-1}-\displaystyle\frac{z-2}{3}\right)=0$\\
则$\mu=1,\lambda=-1$得所求平面$3x+12y+2z-1=0$.
\item 设所求平面为$\mu\left(-5x-y+7\right)+\lambda\left(-x-z+1\right)=0$\\
令其法向量$\mathbf{n}$满足$\mathbf{n}\cdot\left[\left(2,-1,-1\right)\times\left(1,2,-1\right)\right]$,得所求平面方程为$$-3x-y+2z+5=0.$$
\item 同理,设所求平面为$\mu\left(-3x-2y-1\right)+\lambda\left(x-z+1\right)=0$\\令其法向量$\mathbf{n}$满足$\mathbf{n}\cdot\left(3,2,-1\right)$,得所求平面$-x+8y+13z-9=0$.
\end{enumerate}

\item 解:设所求平面为$x-2y+3z+c=0\left(c\neq-4\right).$\\
\begin{enumerate}[(1)]
\item 代入$\left(0,-3,0\right)$得$x-2y+3z-6=0$
\item $d_{O\to\pi}=\displaystyle\frac{\left|c\right|}{\sqrt{1^2+\left(-2\right)^2+3^2}}=1$,得$$x-2y+3z\pm\sqrt{14}=0.$$
\end{enumerate}

\item 解:设所求平面方程为$x+3y+2z=D$\\
其在$\mathbb{X}$轴$\mathbb{Y}$轴$\mathbb{Z}$轴上的截距分别为$D,\displaystyle\frac{D}{3},\displaystyle\frac{D}{2}.$\\
$V=\left|D\right|\cdot\left|\displaystyle\frac{D}{3}\right|\cdot\left|\displaystyle\frac{D}{2}\right|\cdot\displaystyle\frac{1}{3}=6$解得$D=\pm3\sqrt[3]{4}$,则所求平面为$$x+3y+2z\pm3\sqrt[3]{4}=0$$

\item 解:令$x$项系数与$y$项系数相等,即$1+\lambda=3-\lambda$,得平面$2x+2y-2z+9=0$.

\item 解:设所求平面为$\mu\left(x+1\right)+\lambda\left(-3x-2z-6\right)=0$\\令$d_{P\to\pi}=3$,即$\left(5\mu+13\lambda\right)^2=3^2\left[\left(\mu\right)^2+\left(3\lambda\right)^2+\left(2\lambda\right)^2\right]$,解得$\mu=\lambda=\displaystyle\frac{-65\pm18\sqrt{377}}{16}$\\
未完待续...

\item 解:设平面\uppercase\expandafter{\romannumeral1}$:\mu\left(2x-4y+z\right)+\lambda\left(3x-y-9\right)=0$.\\
令\uppercase\expandafter{\romannumeral1}$\perp$\uppercase\expandafter{\romannumeral2}解得\uppercase\expandafter{\romannumeral1}$:x+3y-z-9=0$\\
知所求射影直线方程为$\left\{\begin{array}{l}x+3y-z-9=0\\4x-y+z-1=0\end{array}\right.$

\item 证明:由本书第41页定理2.3.2知要证命题成立,只需证$l_1//l_2$时成立.\\
若$l_1//l_2$此时不妨设$x=\displaystyle\frac{a}{d},y=\displaystyle\frac{b}{d},z=\displaystyle\frac{c}{d}$\\
此时已知方程转化为$A_{i}a+B_{i}b+C_{i}c+D_{i}d=0$\\
当$d=0$时,$l_1//l_2$,此时转化为与定理2.3.2相同得情形,故原命题成立. 
\end{enumerate}

\section{常见曲面}
\subsection{曲面与空间曲线}
\begin{enumerate}
\item 解:整理得$\left(x+1\right)^2+y^2+\left(z-2\right)^2=3^2$,知圆心坐标$\left(-1,0,2\right)$半径$r=3.$

\item 解:\begin{enumerate}[(1)]
\item 方程$r=3.$
\item $\left\{\begin{array}{l}r=\sqrt{x^2+y^2+z^2}=\sqrt{3}\\\theta=\arccos\displaystyle\frac{y}{r}=\mathrm{arccos}\displaystyle\frac{\sqrt{3}}{3}\\\varphi=\mathrm{arccos}\displaystyle\frac{y}{x}=\displaystyle\frac{\pi}{4}\end{array}\right.$,知球面坐标为$\left(r,\varphi,\theta\right)=\left(\sqrt{2},\displaystyle\frac{\pi}{4},\mathrm{arccos}\displaystyle\frac{\sqrt{3}}{3}\right)$\\
$\left\{\begin{array}{l}r=\sqrt{x^2+y^2+z^2}=\sqrt{2}\\\varphi=\arccos\displaystyle\frac{y}{x}=\displaystyle\frac{\pi}{4}\\z=z=1\end{array}\right.$,知球面坐标为$\left(r,\varphi,z\right)=\left(\sqrt{2},\displaystyle\frac{\pi}{4},1\right)$\\
\end{enumerate}

\item 解:\\$\left\{\begin{array}{l}x=a\cos\varphi\\y=a\sin\varphi\\z=z_0\\\varphi=\omega t\\z_0=\upsilon t\end{array}\right.$得参数方程$\left\{\begin{array}{l}x=a\cos\omega t\\y=a\sin\omega t\\z=\upsilon t\end{array}\right.$

\item 解:椭圆曲线.证明略.

\item 解:\begin{enumerate}[(1)]
\item 原点至平面$x+y+z-3=0$的距离$d=3$,则所求圆的半径$r=\sqrt{4-d^2}=1$.易得过原点且与平面$x+y+z-3=0$垂直的直线为$x=y=z$,联立平面方程得圆心坐标$\left(1,1,1\right)$
\item 对原方程组$\left\{\begin{array}{l}x^2+y^2+z^2=5\\x^2+y^2+z^2+x+2y+3z-7=0\end{array}\right.$整理得$\left\{\begin{array}{l}x^2+y^2+z^2=5\\x+2y+3z-2=0\end{array}\right.$\\同理可得半径$r=\displaystyle\frac{4}{7}\sqrt{14}$,圆心坐标$\left(\displaystyle\frac{1}{7},\displaystyle\frac{2}{7},\displaystyle\frac{3}{7}\right)$.
\end{enumerate}

\item 证明:令$t=0$,得坐标$\left(0,0,0\right)$,
令$t=1$,得坐标$\left(\displaystyle\frac{1}{3},\displaystyle\frac{1}{3},\displaystyle\frac{1}{3}\right)$,
令$t=-1$,得坐标$\left(-\displaystyle\frac{1}{3},\displaystyle\frac{1}{3},-\displaystyle\frac{1}{3}\right)$,
设由此三点决定的球面方程为$\pi$,其球心为$\left(x_0,y_0,z_0\right)$
令$\left(\displaystyle\frac{1}{3}-x_0\right)^2+\left(\displaystyle\frac{1}{3}-y_0\right)^2+\left(\displaystyle\frac{1}{3}-z_0\right)^2=\left(-\displaystyle\frac{1}{3}-x_0\right)^2+\left(-\displaystyle\frac{1}{3}-y_0\right)^2+\left(\displaystyle\frac{1}{3}-z_0\right)^2=x_0^2+y_0^2+z_0^2$,\\解得$\left(x_0,y_0,z_0\right)=\left(0,\displaystyle\frac{1}{2},0\right)$,半径$r=\displaystyle\frac{1}{2}$.\\
而$\left(\displaystyle\frac{t}{1+t^2+t^4}\right)^2+\left(\displaystyle\frac{t^2}{1+t^2+t^4}-\displaystyle\frac{1}{2}\right)^2+\left(\displaystyle\frac{t^3}{1+t^2+t^4}\right)^2=\left(\displaystyle\frac{1}{2}\right)^2$\\
则曲线在一球面上,且球面方程为$x^2+\left(y-\displaystyle\frac{1}{2}\right)^2+z^2=\left(\displaystyle\frac{1}{2}\right)^2$

\item 证明:由题,$\left(\sqrt{x^2+y^2}-a\right)^2+z^2=b^2$,则由本书第75页表3.2得\\
面$\left(\sqrt{x^2+y^2}-a\right)^2+z^2=b^2$可由$\left\{\begin{array}{l}\left(y-a\right)^2+z^2=b^2\\x=0\end{array}\right.$绕$\mathbb{Z}$轴旋转得到.故其为一个环面,其环面方程为$\left(y-a\right)^2+z^2=b^2$.

\item 解:$\left\{\begin{array}{l}x=t\\y=1-t\\z=\sqrt{\displaystyle\frac{9}{2}-t^2-\left(1-t\right)^2}\end{array}\right.$

\item 解:$C$到平面$2x-y-3z+11=0$的距离$d=\displaystyle\frac{16}{\sqrt{14}}$,\\故所求方程为$\left(x-3\right)^2+\left(y+5\right)^2+\left(z-2\right)^2=\displaystyle\frac{128}{\sqrt{7}}$
\end{enumerate}

\subsection{柱面与投影曲线}
\begin{enumerate}
\item (1)椭圆柱面,图形略;(2)双曲柱面,图形略;(3)抛物柱面,图形略;(4)抛物柱面,图形略;(5)平面,图形略;(6)两个互相垂直的平面,图形略;(7)两个互相垂直的平面,图形略;(8)三个互相垂直的平面,图形略;

\item 解:\\
(1)$\left\{\begin{array}{l}z^2+4y+4z=0\\x^2+z^2-4z=0\\x^2+4y=0\end{array}\right.$
(2)$\left\{\begin{array}{l}x^2-x+y^2-1=0\\z-x-1=0\\z^2-3z+y^2+1=0\end{array}\right.$
(3)$\left\{\begin{array}{l}x^2-2x-2z^2+6z-3=0\\y-z+1=0\\x^2-2x+y^2+2y+1=0\end{array}\right.$
(4)$\left\{\begin{array}{l}x-z-3=0\\2y+7z-2=0\\7x+2y-18=0\end{array}\right.$
(5)$\left\{\begin{array}{l}z^2+x-1=0\\\left(1-z^2\right)^2+z^2+y^2-1=0\\x^2+y^2-x=0\end{array}\right.$

\item 解:$\left\{\begin{array}{l}x^2+y^2+z^2=1\\x^2+y^2=2z\end{array}\right.$消去$x^2+y^2$得$z^2+2z=1$,解得$z=-1\pm\sqrt{2}$代入原方程中,得$x^2+y^2=-2\pm2\sqrt{2}$,由于$x^2+y^2\geq0$,故$x^2+y^2=-2-2\sqrt{2}$舍去. \\
知曲线方程为$\left\{\begin{array}{l}x^2+y^2=2\left(\sqrt{2}-1\right)\\z=\sqrt{2}-1\end{array}\right.$,其形状为一个半径$r=\sqrt{2\left(\sqrt{2}-1\right)}$的圆.

\item 解:整理方程组得$\left\{\begin{array}{l}\left(z-2\right)^2=4\left(y+1\right)\\x^2=-4y\end{array}\right.$知其图形为一个在$\mathbb{X}O\mathbb{Y}$平面有相同形状的投影曲线的曲线,其两个投影曲线都为抛物线形.\\

\item 解:\begin{enumerate}[(1)]
\item 设$M_0\left(x_0,y_0,z_0\right)$为曲线$\Gamma$上的点,则\\
柱面上的点有$x=x_0-t-1,y=y_0,z=z_0$,且$$\left\{\begin{array}{l}\left(x_0-1\right)^2+\left(y_0+3\right)^2+\left(z_0-2\right)^2+=25\\x_0+y_0-z_0+2=0\end{array}\right.$$消去$x_0,y_0,z_0$得$$\left\{\begin{array}{l}\left(x+t\right)^2+\left(y+3\right)^2+\left(z-2\right)^2+=25\\x+y-z+t+3=0\end{array}\right.\textcircled{1}$$消去参数$t$,得$$\left(z-y-3\right)^2+\left(y+3\right)^2+\left(z-2\right)^2=25$$
\item 设$M_0\left(x_0,y_0,z_0\right)$为曲线$\Gamma$上的点,由直线的方向向量为$\left(0,1,1\right)$得柱面上的点有$x=x_0,=y_0+t,z=z_0+t$,且满足\textcircled{1}式,消去$x_0,y_0,z_0$得$$\left\{\begin{array}{l}\left(x-1\right)^2+\left(y-t+3\right)^2+\left(z-t-2\right)^2+=25\\x+y-z+2=0\end{array}\right.$$无视参数$t$得$$x+y-z+2=0$$.
\item 设$M_0\left(x_0,y_0,z_0\right)$为曲线$\Gamma$上的点,则柱面上的点有$x=x_0-t,y=y_0+t,z=z_0+t$,且$$x_0^2+y_0^2-z_0+1=0\\2x_0^2+2y_0^2+z_0-4=0$$消去参数$t$,得$$\left(x+z-2\right)^2+\left(y-z+2\right)^2-1=0$$.
\item 取准线上的三个点$A\left(0,0,0\right),B\left(2,1,1\right),C\left(2,-1,1\right)$取其母线方向$\mathbf{v}=\lambda\overrightarrow{AB}\times\overrightarrow{AC}=\left(-1,0,2\right),\left(\lambda=\displaystyle\frac{1}{2}\right)$\\
设$M_0\left(x_0,y_0,z_0\right)$在准线上,则对所求柱面上的点有$$x=x_0-t,y=y_0,z=z_0+2t$$且$$x_0=y_0^2+z_0^2=2z_0$$消去$x_0,y_0,z_0$得$$x+t=y^2+\left(z-2t\right)^2=2\left(z-2t\right)$$消去参数$t$,得$$25y^2+\left(z+2x\right)^2-20x-10z=0$$
\end{enumerate}

\item 解:\begin{enumerate}[(1)]
\item 点到轴的距离$d=\displaystyle\frac{\sqrt{41}}{3}$,故圆柱面半径为$r=\displaystyle\frac{\sqrt{41}}{3}$,由定理2.3.2,得圆柱面方程$$2x^2+\displaystyle\frac{5}{4}y^2+\displaystyle\frac{5}{4}z^2+xy-2xz-2yz-2x-\displaystyle\frac{1}{2}y-\displaystyle\frac{1}{2}z-\displaystyle\frac{39}{4}=0$$即$$\left(2y-2z\right)^2+\left(z+2x-1\right)^2+\left(2x+y-1\right)^2=41$$
\item 任取对称轴上的点$\left(x,y,z\right)$到三条母线的距离相等,所以$$\left|\left(x,y,z\right)\times\left(1,1,1\right)\right|=\left|\left(x+1,y,z-1\right)\times\left(1,1,1\right)\right|=\left|\left(x,y+1,z-2\right)\times\left(1,1,1\right)\right|$$化简得对称轴方程为$x=y+2=z-1$,圆柱面上的点到对称轴的距离等于对称轴上的点$\left(0,-2,1\right)$到母线$x=y=z$的距离,故$$\left|\left(0,-2,1\right)\times\left(1,1,1\right)\right|=\left|\left(x,y+2,z-1\right)\times\left(1,1,1\right)\right|$$即$$\left(z-y-3\right)^2+\left(x-z+1\right)^2+\left(y-x+2\right)^2=14$$于是圆周面的方程为$$x^2+y^2+z^2-xy-yz-xz-x+5y-4z=0$$
\item 圆柱面平行于$\mathbb{Z}$轴,其方程明显为$$\left(x-1\right)^2+\left(y-1\right)^2=2$$
\item 圆柱面平行于$\mathbb{Z}$轴,故设所求圆柱面方程为$$\left(x-a\right)^2+\left(y-b\right)^2=r^2$$代入$\left(0,0\right),\left(4,2\right),\left(6,-3\right)$三个点解得$$\left(x-\displaystyle\frac{25}{8}\right)^2+\left(y+\displaystyle\frac{5}{4}\right)^2=\displaystyle\frac{34}{3}$$.
\item 圆柱面平行于$\mathbb{Z}$轴,同理可得$$\left(x+4\right)^2+\left(y-b3\right)^2=25$$
\end{enumerate}
\item 证明:由定理3.2.2,由柱面方程为$$\left(py-nz\right)^2+\left(px-mz\right)^2+\left(nx-my\right)^2=r^2\left(m^2+n^2+p^2\right)$$展开即所得. 
\end{enumerate}

\subsection{锥面和旋转曲面}
\begin{enumerate}
\item 解:\begin{enumerate}[(1)]
\item 任取锥面上一点$\left(x,y,z\right)$,设母线$OM$与准线相交于$M_1\left(x_1,y_1,z_1\right)$,则$$\frac{x_1-0}{x-0}=\frac{y_1-0}{y-0}=\frac{z_1-0}{z-0}$$,令比值为$t$,则$$x_1=xt,y_1=yt,z_1=zt$$.将其代入准线方程并消去参数$t$,整理得所求锥面方程为$$36x^2+9y^2-4z^2=0$$.
\item 同理,所求曲面方程为$$\frac{x^2}{a^2}+\frac{y^2}{b^2}-\frac{z^2}{c^2}=0$$.
\item 同理,所求曲面方程为$$F\left(\frac{hx}{z},\frac{hy}{z}\right)=0$$.
\item 先讨论当$x=y=z$为圆锥曲面的轴时,取圆锥面上一点$\left(x,y,z\right)$,设原点为$O\left(0,0,0\right)$,轴上一点$B\left(a,a,a\right)\left(a\neq0\right)$令$\overrightarrow{OM}\cdot\overrightarrow{OB}=\left|\overrightarrow{OM}\right|\cdot\left|\overrightarrow{OB}\right|=\cos\alpha$,再令$\mathbb{Z}$轴上一点$A\left(0,0,1\right)$为圆锥面上一点,则得以下两个方程:$$\left\{\begin{array}{l}\left(x,y,z\right)\cdot\left(a,a,a\right)=\sqrt{x^2+y^2+z^2}a\sqrt{3}\cos\alpha\\\left(0,0,1\right)\cdot\left(a,a,a\right)=\sqrt{3}a\sqrt{3}\cos\alpha\end{array}\right.$$解得所求圆锥面方程为$$xy+yz+xz=0.$$
同理得其余情况下方程为$$xy-yz+xz3=0$$或$$xy-yz-xz=0$$或$$xy+yz-xz$$.
\item 同理,所求曲面方程为$$5x^2+13y^2+25z^2-10xy-30yz+22xz+4x-4y+4z-4=0$$
\item 同理,所求曲面方程为$$?$$
\item 其轴方程为$x=y=z$,且由题得$$\left(3,2,1\right)\left(1,1,1\right)=\sqrt{3^2+2^2+1^2}\sqrt{1^2+1^2+1^2}\cdot\cos\alpha$$知$\cos^2\alpha=\displaystyle\frac{6}{7}$,由本书第78页第7题讨论,得所求曲面方程为$$\left(x+y+z\right)^2=\frac{6}{7}\left(1^2+1^2+1^2\right)\left(x^2+y^2+z^2\right)$$即$$11x^2+11y^2+11z^2-14xy-14xz-14yz=0$$
\end{enumerate}

\item 解:\begin{enumerate}[(1)]
\item 由命题3.3.1,其曲面为锥面,且是圆锥面,顶点为$\left(0,0,0\right)$
\item 观察方程的形态,利用$x=x'+a$线性变换将其配凑成一个齐次方程,为此,令$$\left\{\begin{array}{l}y+z-x-1=0\\z+x-y-1=0\\x+y-z-1=0\end{array}\right.$$解得$x=y=z=1$,则令$x'=x-1,y'=y-1,z'=z-1$代入原方程整理得$$\left(y'+z'-x'\right)^3=\left(z'+x'-y'\right)^2\left(x'+y'-z'\right)$$由于这样的变换不改变曲面的形状,因此原方程为锥面方程,其顶点为$\left(1,1,1\right)$
\end{enumerate}

\item 证明:整理方程得$$x^2+y^2+z^2-2xy-2yz-2xz=0$$显然其为一个锥面方程且原点是其顶点.再整理得$$\left(x+y+z\right)^2=\left(\frac{\sqrt{6}}{3}\right)^2\left(1^2+1^2+1^2\right)\left(x^2+y^2+z^2\right)$$由本书第7题知,$\sqrt{x}+\sqrt{y}-\sqrt{z}=0$表示一个半顶角为$\mathrm{arccos}\displaystyle\frac{\sqrt{6}}{3}$的圆锥面. 

\item \begin{enumerate}[(1)]
\item $$\frac{\left(y-1\right)^2}{b^2}+\frac{x^2+z^2}{c^2}=1$$
\item $$z-\tan\left(x^2+y^2\right)=0$$
\item $$4\left(x-1\right)^2+y^2+z^2=1$$
\item 设$x=\displaystyle\frac{1}{3}\left(t-2\right),y=\displaystyle\frac{1}{2}\left(t+1\right),z=t$得其旋转曲面方程为$$\left\{\begin{array}{l}x=\sqrt{\displaystyle\frac{\left(t-2\right)^2}{9}+\displaystyle\frac{\left(t+1\right)^2}{4}}\cos\theta\\y=\sqrt{\displaystyle\frac{\left(t-2\right)^2}{9}+\displaystyle\frac{\left(t+1\right)^2}{4}}\sin\theta\\z=t\end{array}\right.$$消去参数得$$x^2+y^2=\displaystyle\frac{\left(z-2\right)^2}{9}+\displaystyle\frac{\left(z+1\right)^2}{4}$$
\item 两直线平行,故得圆柱面方程$$5x^2+5y^2+2z^2+2xy+4xz+4yz+4x-4y-4z-6=0$$
\item 取曲面上一点$M\left(x',y',z'\right)$,则$x'^2+y'^2+z'^2=x^2+y^2+z^2$,有$$\left\{\begin{array}{l}\left(x-x',y-y',z-z'\right)\left(1,2,1\right)=0\\x'^2=y\\x'+z'=0\end{array}\right.$$消去$x',y',z'$得$$3x^2+3y^2-4xy-2xz-4yz-4x-8y-4z=0$$
\end{enumerate}

\item 解:\begin{enumerate}[(1)]
\item 得$\displaystyle\frac{x^2+y^2}{9-\lambda}+\displaystyle\frac{z^2}{4-\lambda}=1$,其形状为(1)椭球面,$\lambda<4$;(2)单页双曲面,$4<\lambda<9$;(3)不存在,$\lambda>9$.
\item 取$\left(x,y,z\right)=\left(at,b,t\right)$,得旋转参数方程为$\left(x,y,z\right)=\left(\sqrt{\left(at\right)^2+b^2}\cos\theta,\sqrt{\left(at\right)^2+b^2}\sin\theta,t\right).$即$$x^2+y^2-a^2z^2-b^2=0$$讨论略.
\end{enumerate}

\item 证明:整理得$$4x^2-9\left(y^2+z^2\right)=36$$可取母线为$\left\{\begin{array}{l}4x^2-9y^2=36\\z=0\end{array}\right.$,转轴为$\mathbb{X}$轴,故其为旋转曲面.

\item 证明:取$l$上一点$\left(m,n,p\right)$,圆周面上一点$\left(x,y,z\right)$,则由于夹角为$\alpha$,得$$\frac{\left(m,n,p\right)\left(x,y,z\right)}{\sqrt{m^2+n^2+p^2}\sqrt{x^2+y^2+z^2}}\cos\alpha$$整理即得所求为$$\left(mx+ny+pz\right)^2=\cos^2\alpha\left(m^2+n^2+p^2\right)\left(x^2+y^2+z^2\right)$$

\item 解:\begin{enumerate}[(1)]
\item 柱面. 说明:对于任一满足方程的点$\left(x_0,y_0,z_0\right)$,都显然会有$\left(x_0+\displaystyle\frac{t}{k},y_0,z_0+t\right)$满足方程,消去$t$,得直线$$\left\{\begin{array}{l}kx-z-kx_0+z_0=0\\y=y_0\end{array}\right.$$知所得点解会张成一条方向固定的直线,又由于本书只讨论代数曲面,故所得点解是连续的,可构成连续的图形,知为柱面. 
\item 锥面. 说明:对任一满足方程的点$\left(x_0,y_0,z_0\right)$,都显然会有$\left(tx_0,ty_0,tz_0\right)$满足方程,又由于是代数曲面,故为锥面. 
\item 柱面.同理$\left(x_0,y_0,z_0\right)$解张为直线解$$x-x_0=\frac{y-y_0}{a}=\frac{z-z_0}{b}$$.
\item 锥面.说明略. 
\end{enumerate}
\end{enumerate}

\subsection{二次曲面}
\begin{enumerate}
\item 解:设所求平面方程为$Ax+Bz=0$,联立椭球面方程得$$\left\{\begin{array}{l}Ax+Bz=0\\\displaystyle\frac{x^2}{a^2}+\displaystyle\frac{y^2}{b^2}+\displaystyle\frac{z^2}{c^2}=1\end{array}\right.$$代入得(1)$A=0$时,显然不满足要求;(2)$A\neq0$时,有$\displaystyle\frac{x^2}{a^2}+\left(\displaystyle\frac{B^2}{A^2b^2}+\displaystyle\frac{1}{c^2}\right)z^2=1$,显然$\displaystyle\frac{1}{a^2}=\displaystyle\frac{B^2}{A^2b^2}+\displaystyle\frac{1}{c^2}$,知所求平面为$$y=\pm z\sqrt{\left(\frac{1}{a^2}-\frac{1}{c^2}\right)b^2}$$
\item 解:由题$$\left(x+4\right)^2+y^2+z^2=4\left[\left(x-2\right)^2+y^2+z^2\right]$$得$$3x^2+3y^2+3z^2-24x=0$$

\item 解:设所求椭球面为$$\displaystyle\frac{x^2}{a^2}+\displaystyle\frac{y^2}{b^2}+\displaystyle\frac{z^2}{c^2}=1$$代入题设得$$\displaystyle\frac{x^2}{9}+\displaystyle\frac{y^2}{16}+\displaystyle\frac{z^2}{36}=1$$

\item 解:$\displaystyle\frac{x^2}{4}=\displaystyle\frac{z^2}{9}$,即$\displaystyle\frac{x}{2}=\pm\displaystyle\frac{z}{3}$,为两条交叉直线.

\item 解:$\left(1-k^2\right)x^2+y^2=1$,形状为(1)圆,$k=0$;(2)椭圆,$-1<k<1$;(3)两条平行直线,$k=\pm1$;(4)双曲线,$\left|k\right|>1$;

\item 解:(1)双曲抛物面,讨论略;(2)$\left\{\begin{array}{l}z=ky\\x=k\end{array}\right.$,利用平行截割法,知为直纹面,讨论略;(3)单页双曲面,讨论略;(4)双叶双曲面,讨论略;

\item 解:$A,B,C>\lambda$时为椭圆面;\\
$A>B>\lambda>C$时为单页双曲面;\\
$A>\lambda>B>C$时为双叶双曲面;\\
$\lambda>A,B,C$时不存在图形. 

\item 解:$\lambda>9$,不存在;$\lambda=9$,不存在;$9>\lambda>4$,双叶双曲面;$\lambda=4$,椭柱面;$4>\lambda>1$,单页双曲面;$\lambda=1$,椭柱面;$\lambda<1$,椭球面.

\item 解:设其方程为$\displaystyle\frac{z^2}{a^2}+\displaystyle\frac{y^2}{b^2}=2x$,代入两点坐标解得$$\frac{3}{5}z^2+\frac{18}{5}y^2=x$$.

\item 解:显然为$x=\pm2$或$y=\pm3$.
\end{enumerate}

\subsection{直纹面}
\begin{enumerate}

\item \begin{enumerate}[(1)]
\item 设单叶双曲面方程为$$\frac{x^2}{a^2}+\frac{y^2}{b^2}-\frac{z^2}{c^2}=1\left(a,b,c\in\mathbb{R^+}\right)$$方程改写为$$\left(\frac{x}{a}-\frac{z}{c}\right)\left(\frac{x}{a}+\frac{z}{c}\right)=\left(1+\frac{y}{b}\right)\left(1-\frac{y}{b}\right)$$
得$u$族直母线$$\left\{\begin{array}{l}\displaystyle\frac{x}{a}-\displaystyle\frac{z}{c}=u\left(1+\displaystyle\frac{y}{b}\right)\\u\left(\displaystyle\frac{x}{a}+\displaystyle\frac{z}{c}\right)=1-\displaystyle\frac{y}{b}\end{array}\right.$$
得$v$族直母线$$\left\{\begin{array}{l}\displaystyle\frac{x}{a}-\frac{z}{c}=v\left(1-\frac{y}{b}\right)\\\displaystyle v\left(\frac{x}{a}+\frac{z}{c}\right)=1+\frac{y}{b}\end{array}\right.$$
对同族中任意两条直母线,其显然不平行,则下证其任意不相交,联立方程得$$\left\{\begin{array}{l}\displaystyle\frac{1}{a}x-\frac{1}{b}u_1y-\frac{1}{c}z-u_1=0\\\displaystyle\frac{u_1}{a}x+\frac{1}{b}y+\frac{u_1}{c}z-1=0\\\displaystyle\frac{1}{a}x-\frac{u_2}{b}y-\frac{1}{c}z-u_2=0\\\displaystyle\frac{u_2}{a}x+\frac{1}{b}y+\frac{u_2}{c}z-1=0\end{array}\right.$$
其判别式$$\left|\begin{array}{cccc}\displaystyle\frac{1}{a}&-\displaystyle\frac{u_1}{b}&-\displaystyle\frac{1}{c}&-u_1\\\displaystyle\frac{u_1}{a}&\displaystyle\frac{1}{b}&\displaystyle\frac{u_1}{c}&-1\\\displaystyle\frac{1}{a}&-\displaystyle\frac{u_2}{b}&\displaystyle-\frac{1}{c}&-u_2\\\displaystyle\frac{u_2}{a}&\displaystyle\frac{1}{b}&\displaystyle\frac{u_2}{c}&-1\end{array}\right|=\displaystyle\frac{1}{a}\displaystyle\frac{1}{b}\displaystyle\frac{1}{c}\left|\begin{array}{cccc}1&-u_1&-1&-u_1\\u_1&1&u_1&-1\\1&-u_2&-1&-u_2\\u_2&1&u_2&-1\end{array}\right|=\displaystyle\frac{4}{abc}u_2\left(u_2-u_1\right)$$
因为其中$u_1\neq u_2$,故其判别式非零,故两条直线异面,同理$v$族异面,(1)得证.
\item 对异族中任意两条直母线,同理,得增广矩阵$$A=\left(\begin{array}{cccc}\displaystyle\frac{1}{a}&\displaystyle-\frac{u}{b}&\displaystyle-\frac{1}{c}&-u\\\displaystyle\frac{u}{a}&\displaystyle\frac{1}{b}&\displaystyle\frac{u}{c}&-1\\\displaystyle\frac{1}{a}&\displaystyle\frac{v}{b}&\displaystyle-\frac{1}{c}&-v\\\displaystyle\frac{v}{a}&\displaystyle-\frac{1}{b}&\displaystyle\frac{v}{c}&-1\end{array}\right)$$
得$$det A=\displaystyle\frac{1}{abc}\left|\begin{array}{cccc}1&-u&1&u\\u&1&-u&1\\1&v&1&v\\v&-1&-v&1\end{array}\right|=0$$
知异族两条直母线共面. \\
$u$族直母线的方向向量$\mathbf{u}=\left(\displaystyle\frac{1}{a},\displaystyle-\frac{u}{b},\displaystyle-\frac{1}{c}\right)\times\left(\displaystyle\frac{u}{a},\displaystyle\frac{1}{b},\displaystyle\frac{u}{c}\right)=\left(\displaystyle\-\frac{u^2}{bc}+\displaystyle\frac{1}{bc},\displaystyle-\frac{2u}{ac},\displaystyle\frac{u^2}{ab}+\displaystyle\frac{1}{ab}\right)$\\
$v$族直母线的方向向量$\mathbf{v}=\left(\displaystyle\frac{1}{a},\displaystyle\frac{v}{b},\displaystyle-\frac{1}{c}\right)\times\left(\displaystyle\frac{v}{a},\displaystyle-\frac{1}{b},\displaystyle\frac{v}{c}\right)=\left(\displaystyle\\\frac{v^2}{bc}-\displaystyle\frac{1}{bc},\displaystyle-\frac{2v}{ac},\displaystyle-\frac{v^2}{ab}-\displaystyle\frac{1}{ab}\right)$\\
令$\mathbf{v}$平行$\mathbf{u}$,则对应数成比例,即$$\displaystyle\frac{-u^2+1}{v^2-1}=\displaystyle\frac{-u}{-v}=\displaystyle\frac{u^2+1}{-\left(v_2+1\right)}$$当且仅当$u=-v$时成立,则$u$族的任一直母线,在$v$族直母线中有且仅有一条直母线与之平行. 
\end{enumerate}

\item 证明:由本书第84页定理3.5.1证法2得\\
得$u$族直母线:$\left\{\begin{array}{l}\displaystyle\frac{x}{a}+\displaystyle\frac{y}{b}=2u\\u\left(\displaystyle\frac{x}{a}-\displaystyle\frac{y}{b}\right)=z\end{array}\right.$
$v$族直母线:$\left\{\begin{array}{l}\displaystyle\frac{x}{a}-\displaystyle\frac{y}{b}=2v\\v\left(\displaystyle\frac{x}{a}+\displaystyle\frac{y}{b}\right)=z\end{array}\right.$\\
与1中同理,联立方程其增广矩阵
$$A=\left(\begin{array}{cccc}\displaystyle\frac{1}{a}&\displaystyle\frac{1}{b}&0&-2u_1\\\displaystyle\frac{u_1}{a}&\displaystyle\frac{u_1}{b}&-1&0\\\displaystyle\frac{1}{a}&\displaystyle\frac{1}{b}&0&-2u_2\\\displaystyle\frac{u_2}{a}&\displaystyle-\frac{u_2}{b}&-1&0\end{array}\right)$$
知$$det A=\displaystyle\frac{-2}{ab}\left|\begin{array}{cccc}1&1&0&u_1\\u_1&-u_1&-1&0\\1&1&0&u_2\\u_2&-u_2&-1&0\end{array}\right|=\displaystyle-\frac{4}{ab}\left(u_1-u_2\right)^2,\left(u_1\neq u_2\right)$$
知同族的任意两条直母线异面($v$族同理)\\
联立任意两条异族直母线方程,同理其增广矩阵为
$$A=\left(\begin{array}{cccc}\displaystyle\frac{1}{a}&\displaystyle\frac{1}{b}&0&-2u\\\displaystyle\frac{u}{a}&\displaystyle-\frac{u}{b}&-1&0\\\displaystyle\frac{1}{a}&\displaystyle-\frac{1}{b}&0&-2v\\\displaystyle\frac{v}{a}&\displaystyle\frac{v}{b}&-1&0\end{array}\right)$$
知$$det A=\displaystyle\frac{2}{ab}\left|\begin{array}{cccc}1&1&0&u\\u&-u&1&0\\1&-1&0&v\\v&v&1&0\end{array}\right|=0$$
知异族的任意两条直母线相交.\\
$u$族直母线的方向向量$\mathbf{u}=\left(\displaystyle\frac{1}{a},\displaystyle\frac{1}{b},0\right)\times\left(\displaystyle\frac{u}{a},\displaystyle\frac{u}{b},-1\right)=\left(\displaystyle-\frac{1}{b},\displaystyle\frac{1}{a},\displaystyle-\frac{2u}{ab}\right)$\\
知其平行于平面$\displaystyle\frac{1}{a}x+\displaystyle\frac{1}{b}y=0$,对$v$族同理.\\
知同族的全体直母线平行于同一平面. 

\item (1)$\left\{\begin{array}{l}x+z=uy\\-y=u\left(x-z\right)\end{array}\right.$或$\left\{\begin{array}{l}x+z=-vy\\y=v\left(x-z\right)\end{array}\right.$\\
(2)$\left\{\begin{array}{l}1=aux\\y=uz\left(x-z\right)\end{array}\right.$或$\left\{\begin{array}{l}1=vay\\x=vz\left(x-z\right)\end{array}\right.$

\item 证明:\begin{enumerate}[(1)]
\item $$\left(x+z\right)^2=\left(1+y\right)\left(1-y\right)$$
其直母线族方程为$$\left\{\begin{array}{l}x+z=u\left(1+y\right)\\u\left(x+z\right)=1-y\end{array}\right.$$
其方向向量$\left(1,-u,1\right)\times\left(u,1,u\right)=\left(u^2-1,0,1+u^2\right)=\left(u^2+1\right)\left(-1,0,1\right)//\left(1,-1,1\right)$\\
方向为定向,知为柱面. 
\item $$\left(y+z\right)^2=\left(1+x\right)\left(1-x\right)$$
得$$\left\{\begin{array}{l}y+z=u\left(1+x\right)\\u\left(y+z\right)=1-x\end{array}\right.$$
其方向向量$\left(u,-1,-1\right)\times\left(1,u,u\right)=\left(0,-1-u^2,1+u^2\right)//\left(0,-1,1\right)$\\
方向为定向,知为柱面. 
\item $$\left\{\begin{array}{l}x+y=u\left(x+2y+z\right)\\u\left(y+z\right)=1\end{array}\right.$$
其方向向量$\left(1-u,1-2u,-u\right)\times\left(0,u,u\right)=\left(-u^2+u,u^2-u,u-u^2\right)//\left(-1,1,-1\right)$\\
知为柱面. 
\item $$\left\{\begin{array}{l}x+y+z=u\left(x-y-z\right)\\u\left(x+y+z\right)=x-y-z\end{array}\right.$$
其方向向量$\left(1-u,1+u,1+u\right)\times\left(u-1,u+1,u+1\right)=\left(0,-2\left(1+u\right),2\left(1+u\right)\right)//\left(0,-1,1\right)$\\
知为柱面. 
\end{enumerate}

\item 解:\begin{enumerate}[(1)]
\item 其直母线方程为$\left\{\begin{array}{l}x+2y=16u\\u\left(x-2y\right)=z\end{array}\right.$或$\left\{\begin{array}{l}x+2y=vz\\v\left(x-2y\right)=16\end{array}\right.$\\
其方向向量$\left(1,2,0\right)\times\left(u,-2u,-1\right)\cdot\left(3,2,-4\right)=0$\\
得直线方程$\displaystyle\frac{x}{-2}=\displaystyle\frac{y-2}{1}=\displaystyle\frac{z+1}{-1}$\\
同理解得另一条直线$\displaystyle\frac{x}{2}=\displaystyle\frac{y+4}{1}=\displaystyle\frac{z+4}{2}$\\
\item $\displaystyle\frac{x}{-2}=\displaystyle\frac{y-16}{1}=\displaystyle\frac{z+32}{-4}$或$\displaystyle\frac{x}{2}=\displaystyle\frac{y+32}{1}=\displaystyle\frac{z+128}{8}$
\item $\displaystyle\frac{x-2}{2}=\displaystyle\frac{y-3}{3}=\displaystyle\frac{z}{0}$或$\displaystyle\frac{x}{2}=\displaystyle\frac{y-6}{-3}=\displaystyle\frac{z+4}{4}$
\end{enumerate}
\item 解:\begin{enumerate}[(1)]
\item $\displaystyle\frac{x-6}{3}=\displaystyle\frac{y-2}{0}=\displaystyle\frac{z-8}{4}$或$\displaystyle\frac{x-3}{0}=\displaystyle\frac{y-1}{1}=\displaystyle\frac{z-2}{2}$
\item $\theta=\mathrm{arccos}\displaystyle\frac{8}{25}\sqrt{5}$
\end{enumerate}

\item 略.

\item 解:Cayley三次直纹面可整理为$$\frac{x^2}{1-z}+\frac{y^2}{1+z}=1$$故其参数方程可写为$$\left\{\begin{array}{l}x=\sqrt{1-u}\cos\theta\\y=\sqrt{1+u}\sin\theta\\z=u\end{array}\right.$$

\item 证明:$u$族直母线族为$$\left\{\begin{array}{l}x+3z=u\left(1+y\right)\\u\left(x-3z\right)=1-y\\z=u\end{array}\right.$$消去$z$,得直线在$\mathbb{X}\mathbb{O}\mathbb{Y}$平面上的投影为
$$2ux+\left(1-u^2\right)y=u^2+1$$
即$$\displaystyle\frac{2u}{1+u^2}x+\displaystyle\frac{1-u^2}{1+u^2}y=1\textcircled{1}$$
又其腰椭圆为$$x^2+y^2=1\textcircled{2}$$
\textcircled{1}中$$\left(\displaystyle\frac{2u}{1+u^2}\right)^2+\left(\displaystyle\frac{1-u^2}{1+u^2}\right)^2=1$$
知\textcircled{1}是\textcircled{2}的切线,证毕.
\end{enumerate}

\section{二次曲面的分类}
\subsection{坐标变换}
\begin{enumerate}
\item 解:由题得
$$\left(\begin{array}{c}\mathbf{e_1}'\\\mathbf{e_2}'\\\mathbf{e_3}'\end{array}\right)=\left(\begin{array}{ccc}\displaystyle-\frac{1}{3}&\displaystyle\frac{2}{3}&\displaystyle\frac{2}{3}\\\displaystyle\frac{2}{3}&\displaystyle-\frac{1}{3}&\displaystyle\frac{2}{3}\\\displaystyle\frac{2}{3}&\displaystyle\frac{2}{3}&\displaystyle-\frac{1}{3}\end{array}\right)\left(\begin{array}{c}\mathbf{e_1}\\\mathbf{e_2}\\\mathbf{e_3}\end{array}\right)$$
知转轴公式为
$$\left(\begin{array}{c}x\\y\\z\end{array}\right)=\left(\begin{array}{ccc}\displaystyle-\frac{1}{3}&\displaystyle\frac{2}{3}&\displaystyle\frac{2}{3}\\\displaystyle\frac{2}{3}&\displaystyle-\frac{1}{3}&\displaystyle\frac{2}{3}\\\displaystyle\frac{2}{3}&\displaystyle\frac{2}{3}&\displaystyle-\frac{1}{3}\end{array}\right)\left(\begin{array}{c}x'\\y'\\z'\end{array}\right)$$
$M$在新系下的坐标$M'=\left(1,-1,0\right)$

\item 解:取三个方向向量
$$\mathbf{n_1}=\left(1,1,1\right),\mathbf{n_2}=\left(1,-2,1\right),\mathbf{n_3}=\left(1,0,-1\right)$$
易知$\mathbf{n_1}\perp\mathbf{n_2}\perp\mathbf{n_3}$,且$\left(\mathbf{n_1},\mathbf{n_2},\mathbf{n_3}\right)=4>0$,即
$$\left(\begin{array}{c}\mathbf{e_1}'\\\mathbf{e_2}'\\\mathbf{e_3}'\end{array}\right)=\left(\begin{array}{ccc}\displaystyle\frac{\sqrt{3}}{3}&\displaystyle\frac{\sqrt{3}}{3}&\displaystyle\frac{\sqrt{3}}{3}\\\displaystyle\frac{\sqrt{6}}{6}&\displaystyle-\frac{\sqrt{6}}{6}&\displaystyle\frac{\sqrt{6}}{6}\\\displaystyle\frac{1}{\sqrt{2}}&0&\displaystyle-\frac{1}{\sqrt{2}}\end{array}\right)\left(\begin{array}{c}\mathbf{e_1}\\\mathbf{e_2}\\\mathbf{e_3}\end{array}\right)$$
并利用三条直线的公共点$(0,0,0)$作为新原点,得坐标变换公式
$$\left(\begin{array}{c}x\\y\\z\end{array}\right)=\left(\begin{array}{ccc}\displaystyle\frac{\sqrt{3}}{3}&\displaystyle\frac{\sqrt{6}}{6}&\displaystyle\frac{1}{\sqrt{2}}\\\displaystyle\frac{\sqrt{3}}{3}&\displaystyle-\frac{\sqrt{6}}{3}&0\\\displaystyle\frac{\sqrt{3}}{3}&\displaystyle\frac{\sqrt{6}}{6}&\displaystyle-\frac{1}{\sqrt{2}}\end{array}\right)\left(\begin{array}{c}x'\\y'\\z'\end{array}\right)$$
\item 解:取$\prod_1\bigcap\prod_2\bigcap\prod_3=\left(\displaystyle-\frac{1}{2},1,\displaystyle\frac{1}{2}\right)$为原点.\\
$\mathbf{n_1^0}=\pm\left(\displaystyle\frac{1}{\sqrt{3}},\displaystyle\frac{1}{\sqrt{3}},\displaystyle\frac{1}{\sqrt{3}}\right)$为$\mathbb{X}$轴方向;
$\mathbf{n_2^0}=\pm\left(\displaystyle\frac{1}{\sqrt{2}},0,\displaystyle-\frac{1}{\sqrt{2}}\right)$为$\mathbb{Y}$轴方向;
$\mathbf{n_3^0}=\pm\left(\displaystyle\frac{1}{\sqrt{6}},\displaystyle-\frac{2}{\sqrt{6}},\displaystyle\frac{1}{\sqrt{6}}\right)$为$\mathbb{Z}$轴方向;\\
控制正负号使原点落在新坐标系的第一卦限,有坐标变换公式为
$$\left(\begin{array}{c}x\\y\\z\end{array}\right)=\left(\begin{array}{ccc}\pm\mathbf{n_1^0}^{\mathbf{T}}&\pm\mathbf{n_2^0}^{\mathbf{T}}&\pm\mathbf{n_3^0}^{\mathbf{T}}\end{array}\right)\left(\begin{array}{c}x'\\y'\\z'\end{array}\right)+\left(\begin{array}{c}-\displaystyle\frac{1}{2}\\1\\\displaystyle\frac{1}{2}\end{array}\right)\textcircled{1}$$
由于过渡矩阵取的是正交矩阵,所以\textcircled{1}即为:
$$\left(\begin{array}{c}x'\\y'\\z'\end{array}\right)=\left(\begin{array}{c}\pm\mathbf{n_1^0}\\\pm\mathbf{n_2^0}\\\pm\mathbf{n_3^0}\end{array}\right)\left(\begin{array}{c}x+\displaystyle\frac{1}{2}\\y-1\\z-\displaystyle\frac{1}{2}\end{array}\right)$$代入$\left(0,0,0\right)$得
$$\left(\begin{array}{l}x'\\y'\\z'\end{array}\right)=\left(\begin{array}{l}\mp\displaystyle\frac{1}{\sqrt{3}}\\\mp\displaystyle\frac{1}{\sqrt{2}}\\\mp\displaystyle\frac{2}{\sqrt{6}}\end{array}\right)$$
知取$-\mathbf{n_1^0}$为$\mathbb{X}$轴正方向单位向量,取$-\mathbf{n_1^0}$为$\mathbb{X}$轴正方向单位向量,取$+\mathbf{n_2^0}$为$\mathbb{Y}$轴正方向单位向量,取$+\mathbf{n_3^0}$为$\mathbb{Z}$轴正方向单位向量.同时$\left(\mathbf{n_1^0},\mathbf{n_2^0},\mathbf{n_3^0}\right)>0$知为右手系得坐标变换公式
$$\left(\begin{array}{c}x\\y\\z\end{array}\right)=\left(\begin{array}{ccc}\displaystyle-\frac{1}{\sqrt{3}}&\displaystyle\frac{1}{\sqrt{2}}&\displaystyle\frac{1}{\sqrt{6}}\\-\displaystyle\frac{1}{\sqrt{3}}&0&-\displaystyle\frac{2}{\sqrt{6}}\\-\displaystyle\frac{1}{\sqrt{3}}&-\displaystyle\frac{1}{\sqrt{2}}&\displaystyle\frac{1}{\sqrt{6}}\end{array}\right)\left(\begin{array}{c}x'\\y'\\z'\end{array}\right)+\left(\begin{array}{c}\displaystyle\frac{1}{2}\\-1\\-\displaystyle\frac{1}{2}\end{array}\right)$$

\item D. 解:$x^2+y^2=2xy$即$\left(x-y\right)^2=0$.

\item 可以这样:
$$\left(\begin{array}{c}x'\\y'\\z'\end{array}\right)=\left(\begin{array}{ccc}2&3&4\\0&0&0\\0&0&0\end{array}\right)\left(\begin{array}{c}x\\y\\z\end{array}\right)+\left(\begin{array}{c}5\\0\\0\end{array}\right)$$
另外的高等解法:视$2x+3y+4z+5=0$为二次方程
$$\left(2x+3y+4z+5\right)^2=4x^2+9y^2+16z^2+12xy+16xz+24yz+20x+30y+40z+25=0$$
特征方程为$\lambda^3-29\lambda^2=0$,取变换公式为
$$\left(\begin{array}{c}x\\y\\z\end{array}\right)=\left(\begin{array}{ccc}\displaystyle\frac{2}{\sqrt{29}}&\displaystyle\frac{3}{\sqrt{22}}&\displaystyle-\frac{17}{\sqrt{638}}\\\displaystyle\frac{3}{\sqrt{29}}&\displaystyle\frac{2}{\sqrt{22}}&-\displaystyle\frac{18}{\sqrt{638}}\\\displaystyle\frac{4}{\sqrt{29}}&\displaystyle\frac{3}{\sqrt{22}}&\displaystyle\frac{5}{\sqrt{638}}\end{array}\right)\left(\begin{array}{c}x'\\y'\\z'\end{array}\right)+\left(\begin{array}{c}0\\0\\5\end{array}\right)$$

\item 其特征方程为$$\lambda^3-625\lambda=0$$,解得$$\lambda_1=0,\lambda_2=\lambda_3=25$$且其为无心二次曲面,知为椭圆抛物面. 

\item 设$\mathbf{e_1}'=\left(\cos\theta\sin\varphi,\sin\theta,\cos\theta\cos\varphi\right)$,令
$$\mathbf{e_1}'\cdot\left(1,1,1\right)=\left(0,0,1\right)\left(1,1,1\right)$$
且
$$\displaystyle\frac{\left[\mathbf{e_1}'\left(1,1,1\right)\right]\cdot\left[\left(0,0,1\right)\times\left(1,1,1\right)\right]}{\sqrt{3}\cdot\sqrt{3}\cdot1\cdot1}=\cos\displaystyle\frac{2}{3}\pi$$
得$\mathbf{e_1}'=\left(\displaystyle\frac{2}{3},\displaystyle\frac{1}{3},\displaystyle\frac{2}{3}\right)$,同理得$\mathbf{e_2}',\mathbf{e_3}'$,知坐标变换公式为
$$\left(\begin{array}{c}x\\y\\z\end{array}\right)=\left(\begin{array}{ccc}\displaystyle\frac{1}{3}&\displaystyle-\frac{1}{3}&\displaystyle\frac{2}{3}\\\displaystyle\frac{2}{3}&\displaystyle-\frac{2}{3}&-\displaystyle\frac{1}{3}\\\displaystyle-\frac{1}{3}&\displaystyle\frac{2}{3}&\displaystyle\frac{2}{3}\end{array}\right)\left(\begin{array}{c}x'\\y'\\z'\end{array}\right)$$

\end{enumerate}

\subsection{二次曲面的渐进方向和中心}
\begin{enumerate}
\item 由定理4.2.1得二次曲面中心分别为$(-3,1,-2)$和$z=0$.

\item B.解:令$F_1(x,y,z)=F_2(x,y,z)=F_3(x,y,z)=0$,得
$$\left\{\begin{array}{l}x=0\\-y-1=0\\-0.5=0\end{array}\right.$$无解. 知为无心二次曲面. 

\item 解:\begin{enumerate}[(1)]
\item 由定理4.2.1,$I_3=\left|\begin{array}{ccc}3&-2&1\\-2&5&-1\\1&-1&3\end{array}\right|=29$知为中心二次曲面.
\item 由定理4.2.1,$I_3=\left|\begin{array}{ccc}4&0&0\\0&-1&1\\0&1&-1\end{array}\right|=0$知为非中心二次曲面.其线性方程组为
$$\left\{\begin{array}{l}3x-2y+z+1=0\\-2x+5y-z+6=0\\x-y+3z+5=0\end{array}\right.$$
对系数矩阵$A^*$的秩$r$和增广矩阵$B$的秩$R$,易得$r\neq R$,为无心二次曲面.
\end{enumerate}

\item 解:\begin{enumerate}[(1)]
\item 其中心坐标为$(0,0,0)$,知其渐进锥面方程为
$$\phi(x-0,y-0,z-0)=y^2-2z^2+2xz=0$$.
\item 其中心坐标为$(0,0,0)$,知其渐进锥面方程为
$$x^2+y^2+z^2-4xy-4yz-4xz=0$$.
\item $I_3=0$,知为非中心二次曲面,知(可能)不存在渐进锥面/
\end{enumerate}

\item 解:(1)$x=y=1$;(2)$x=2y=2$;(3)没有中心;

\item 解:(1)$2x-y+3z+2=0$;(2)$x=\displaystyle\frac{3}{2}y=3(z-2)$

\item 解:对锥面$ax^2+by^2+cz^2=0$,有
$$XF_1(x_0,y_0,z_0)+YF_2(x_0,y_0,z_0)+ZF_3(x_0,y_0,z_0)=aXx_0+bYy_0+cZz_0=0$$时,直线于二次曲面相切. \\
讨论:\textcircled{1}$\phi(X,Y,Z)=aX^2+bY^2+cZ^2\neq0$时,直线与二次曲面相切;\textcircled{2}$\phi(X,Y,Z)=0$时,因为恒有$F(x_0,y_0,z_0)=0$,知直线在二次曲面上. 

\item 解:将直线代入方程,整理得$$7t^2-7t+1=0$$解得$$t_1=t_2=\frac{7\pm\sqrt{21}}{14}$$得两个交点为$\left(\displaystyle\frac{-7\pm2\sqrt{21}}{7},\displaystyle\frac{7\pm2\sqrt{21}}{7},0\right)$

\item 解:将直线代入曲面$Delta$方程,整理得$$4t-1=0$$解得$$t=\frac{1}{4}$$得交点$(2,\displaystyle\frac{1}{2},0)$\\
证明:直线$l$的方向向量为$(4,2,0)$代入$\phi(X,Y,Z)$得$\phi(X,Y,Z)=0$,令
$$XF_1(x_0,y_0,z_0)+YF_2(x_0,y_0,z_0)+ZF_3(x_0,y_0,z_0)=F(x_0,y_0,z_0)=0$$\\
即$$\left\{\begin{array}{l}4(x_0-y_0+2z_0+\displaystyle\frac{3}{2})+2(-x_0-z_0)=0\textcircled{1}\\x_0^2+z_0^2-2x_0y_0-y_0z_0+4z_0x_0+3x_0-5z_0=0\textcircled{2}\end{array}\right.$$
\textcircled{1}代入\textcircled{2}中整理得
$$z_0(y_0-2z_0-8)=0$$知$z_0=0$或$y_0-2z_0-8=0$.
与\textcircled{1}联立方程解得
$z_0=x_0-13=\displaystyle\frac{y_0-8}{2}$或$x_0-2y_0+3=z_0=0$,知存在且存在无数条所求直线,取$(-3,0,0)$点得其中一条直线为$$\displaystyle\frac{x+3}{4}=\displaystyle\frac{y}{2}=\displaystyle\frac{z}{0}$$
\end{enumerate}
\subsection{二次曲面的对称面和主径面}
\begin{enumerate}

\item

\end{enumerate}

\subsection{二次曲面的化简与分类}
\begin{enumerate}

\item

\end{enumerate}

\subsection{二次曲面的切线与切平面}
\begin{enumerate}

\item

\end{enumerate}

\bibliography{math}

\end{document}